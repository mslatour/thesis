%%%%%%%%%%%%%%%%%%%%
% Chapter: Results %
%%%%%%%%%%%%%%%%%%%%
In this chapter the result of the experiment are enumerated and discussed.
Table~\ref{tab:exp_stats_students} shows the student statistics for each
lesson. Although the results will be analyzed per population, most of the
visualizations are grouped together. These visualizations will first be
explained by Section~\ref{sec:results_explanation}.
%Section~\ref{sec:results_general} then lists some general statistics of the
%experiment.
Sections~\ref{sec:results_rules}~through~\ref{sec:results_nimsum}
analyse the results per lesson.

%\begin{itemize}
%	\item Add description of each visualisation and general statistics to main
%		part and not in section?
%\end{itemize}

\section{Description of each visualization}
\label{sec:results_explanation}
The experiment results are captured in different views. This section describes
what each figure and table means.
\paragraph{Three curves of cumulative regret}
Figure~\ref{fig:exp_cumul1} and Figure~\ref{fig:exp_cumul2} show the cumulative
regret curves per population. Each subfigure contains three curves, which are
labelled \emph{theoretical best}, \emph{population best} and \emph{generation
best}. The curve labelled \emph{theoretical best} shows the cumulative regret
compared to the perfect score of a normalized learning gain of 1. This score
could only be achieved if all evaluations of a particular sequence returned the
maximum result of 1, something which is not likely in a noisy environment. That
is why two more curves have been plotted. The curve labelled \emph{population
best} plots the cumulative regret compared to the estimated fitness of the best
sequence ever evaluated by the \emph{TutOER} system in that population.
Similarly, the \emph{generation best} labelled curve shows the cumulative
regret compared to the estimated fitness of the best sequence available in the
current generation. For all fitness values, the estimated value at the end of
the experiment was used while calculating the regret. The reason for that is
that comparing with the estimated values at the moment of the decision is less
interesting. Theoretical results from the literature about genetic
algorithm and UCB already tell us how they will deal with exploration versus
exploitation and the regret that would be caused by that. Furthermore from the
simulations performed in this thesis it appears that the algorithm is properly
implemented. What the experiment is intended to find out is whether the task of
OER quality assessment by curriculum sequencing is feasible in a realistic
scenario with noise due to real students. For that purpose it is more
interesting to see whether the \emph{TutOER} system would temporarily discard
optimal sequences that have seem to perform poorly at the time, or whether the
system is robust enough.
\paragraph{Fitness variance}
Figures \ref{fig:exp_eval_rules}, \ref{fig:exp_eval_intuition},
\ref{fig:exp_eval_binary} and \ref{fig:exp_eval_nimsum} show the observed
evaluations of the best sequence in each population. Each figure contains a
blue and a green line. The blue line connects the individual evaluation resuls.
The green line shows the estimated fitness after each evaluation.
\paragraph{Sorted number of evaluations}
Figure~\ref{fig:exp_ages1} and Figure~\ref{fig:exp_ages2} show the division of
each evaluation over all evaluated sequences. The sequences are sorted, such
that the most evaluated sequence comes first.
\paragraph{Coverage curve}
Figure~\ref{fig:exp_cover1} and Figure~\ref{fig:exp_cover2} show the percentage
of sequences evaluated after each evaluation. The total number of possible
sequences is 40. That means each new encountered sequence adds 2.5\% to the
coverage. There are limits to what percentages can theoretically be achieved in
each generation. Each population is initialized with four different sequences
in its first generation. Furthermore, there are seven individuals in each
consecutive generation. That means that the maximum coverage is given by
$4+7*(i-1)$, for each $i$th generation.
\paragraph{Diversity table}
Table~\ref{tab:results_diversity} shows the percentage of unique sequences at
each generation for each population. A diversity of 43\% indicates that across
the seven individuals three distinct chromosomes were divided. Similarly, a
diversity of 14\% means that all individuals have the same chromosome. The
table does not capture how many occurences each chromosome has. This is also
not relevant. The UCB-1 algorithm selects which sequence to evaluate regardless
of their number of occurences in the generation.
%\section{General statistics}
%\label{sec:results_general}
%Table~\ref{tab:exp_stats_students} shows the student statistics for each
%lesson.
%\begin{itemize}
%	\item link to diversity table
%	\item discuss exam results (nim endgame)
%\end{itemize}
\begin{table}
	\centering
	\caption[Student statistics of each lesson]{The table shows five student statistics for each lesson.
	First, the number of students that completed the \emph{low} category.
	Second, the number of bootstrap values that were used in the
	\emph{low} category. Third, the number of students that completed the
	\emph{high} category. Fourth, the number of bootstrap values that were used in
	the \emph{high} category. Last, the number of students that skipped the
	lesson.}
	\label{tab:exp_stats_students}
	\begin{tabular}{l|lllll}\hline
		\multicolumn{1}{l}{}& \multicolumn{2}{c}{Category \emph{low}} & \multicolumn{2}{c}{Category
			\emph{high}} & \\
		\multicolumn{1}{l}{Lesson} & Completed & Bootstrapped & Completed & Bootstrapped & Skipped\\
		\hline
		Rules & 204 & 19 & 19 & 10 & 14 \\
		Intuition & 85 &  23 & 71 & 30 & 81 \\
		Binary & 106 & 26 & 16 & 11 & 117 \\
		Nim-sum & 154 & 12 & 13 & 9 & 6 \\
	\end{tabular}
\end{table}

\begin{table}
	\caption{Diversity percentage at each generation}
	\label{tab:results_diversity}
	\begin{subtable}{\linewidth}
	\centering
	\caption{generation 1-12}
		\begin{tabular}{l|llllllllllll}\hline
		\multicolumn{1}{l}{} & \multicolumn{12}{c}{Generation} \\
		\multicolumn{1}{l}{Population} & 1 & 2 & 3 & 4 & 5 & 6 & 7 & 8 & 9 & 10 & 11 & 12\\\hline
		Rules Low & 57\% & 57\% & 43\% & 57\% & 43\% & 29\% & 14\% & 14\% & 29\% & 14\% & 29\% & 29\% \\
		Rules High & 57\% & 29\% & 29\% & \textemdash & \textemdash & \textemdash & \textemdash & \textemdash & \textemdash & \textemdash & \textemdash & \textemdash \\
		Intuition Low & 57\% & 43\% & 29\% & 29\% & 29\% & 14\% & 29\% & 14\% & 14\% & 14\% & 29\% & \textemdash \\
		Intuition High & 57\% & 14\% & 14\% & 14\% & 29\% & 29\% & 14\% & 29\% & 29\% & 14\% & 29\% & \textemdash \\
		Binary Low & 57\% & 57\% & 43\% & 86\% & 29\% & 14\% & 43\% & 43\% & 29\% & 29\% & 43\% & 14\% \\
		Binary High & 57\% & 43\% & 29\% & \textemdash & \textemdash & \textemdash & \textemdash & \textemdash & \textemdash & \textemdash & \textemdash & \textemdash \\
		NimSum Low & 57\% & 57\% & 29\% & 14\% & 14\% & 14\% & 14\% & 14\% & 29\% & 14\% & 14\% & 29\% \\
		NimSum High & 57\% & 86\% & 86\% & \textemdash & \textemdash & \textemdash & \textemdash & \textemdash & \textemdash & \textemdash & \textemdash & \textemdash \\
		\end{tabular}
	\end{subtable}

	\medskip
	\begin{subtable}{\linewidth}
	\centering
	\caption{generation 13-24}
		\begin{tabular}{l|llllllllllll}\hline
		\multicolumn{1}{l}{} & \multicolumn{12}{c}{Generation} \\
		\multicolumn{1}{l}{Population} & 13 & 14 & 15 & 16 & 17 & 18 & 19 & 20 & 21 & 22 & 23 & 24\\\hline
		Rules Low & 43\% & 29\% & 29\% & 29\% & 29\% & 29\% & 29\% & 14\% & 14\% & 29\% & 43\% & \textemdash \\
		Rules High & \textemdash & \textemdash & \textemdash & \textemdash & \textemdash & \textemdash & \textemdash & \textemdash & \textemdash & \textemdash & \textemdash & \textemdash \\
		Intuition Low & \textemdash & \textemdash & \textemdash & \textemdash & \textemdash & \textemdash & \textemdash & \textemdash & \textemdash & \textemdash & \textemdash & \textemdash \\
		Intuition High & \textemdash & \textemdash & \textemdash & \textemdash & \textemdash & \textemdash & \textemdash & \textemdash & \textemdash & \textemdash & \textemdash & \textemdash \\
		Binary Low & 29\% & 14\% & \textemdash & \textemdash & \textemdash & \textemdash & \textemdash & \textemdash & \textemdash & \textemdash & \textemdash & \textemdash \\
		Binary High & \textemdash & \textemdash & \textemdash & \textemdash & \textemdash & \textemdash & \textemdash & \textemdash & \textemdash & \textemdash & \textemdash & \textemdash \\
		NimSum Low & 14\% & 14\% & 14\% & 14\% & 14\% & \textemdash & \textemdash & \textemdash & \textemdash & \textemdash & \textemdash & \textemdash \\
		NimSum High & \textemdash & \textemdash & \textemdash & \textemdash & \textemdash & \textemdash & \textemdash & \textemdash & \textemdash & \textemdash & \textemdash & \textemdash \\
		\end{tabular}
	\end{subtable}
\end{table}
\section{Lesson: Rules}
\label{sec:results_rules}
The \emph{Rules low} population completed 22 generations and received 
223 evaluations by students. At the end of the experiment
this population converged to the best sequence encountered. The
sequence contains Resource~3 and Resource~1, in that order, and will be
referred to as \emph{Sequence~[3,1]} for the rest of the chapter. In total 103
out of 178 students that got \emph{Sequence~[3,1]} answered all post-test
questions correctly. A graph of all evaluations of this sequence can be seen in
Figure~\ref{fig:exp_eval_rules_low_372}.
The estimated fitness of \emph{Sequence~[3,1]} at the end of the experiment was
0.7634. \emph{Sequence~[3,1]} was created by a mutation on Resource~3 in the transition
from the fourth to the fifth generation that added Resource~1. At first,
Resource~1 dissapeared from the population after not being selected for the
second generation. Resource~3 had the highest fitness of the first four
generations \\\\
\noindent
Figure~\ref{fig:exp_cumul_rules_low} shows that after twenty evaluations the
sequences selected by the \emph{TutOER} system turn out to be the best of each
generation. After 40 evaluations, in the fifth generation, \emph{TutOER}
switches to the evaluation of \emph{Sequence~[3,1]}. This is only deviated from at
two points. First, at the 81st evaluation, the \emph{TutOER} system tried out
Resource~1 once more. Second, from evaluation 141 till 144 the \emph{TutOER}
system evaluated the mirrored version of \emph{Sequence~[3,1]} that ended with
Resource~3. In both cases however, the \emph{TutOER} system returned to the
optimal sequence.\\\\
\noindent
Figure~\ref{fig:exp_cover_rules_low} shows the \emph{Rules low} population
encountered a fifth of all sequences, namely eight out of 40. Although this was
enough to find \emph{Sequence~[3,1]}, the large majority of sequences was not
attempted. No sequences of three resources were attempted. These sequences
filled 60\% of the total number of possible sequences in this experiment. When
comparing to the possible sequences of two resources maximum, the \emph{TutOER}
system evaluated half.\\\\
\noindent
The \emph{Rules high} population encountered five out of 40 possible sequences.
However, in contrast to \emph{Rules low}, only 30 evaluations were collected
As a result, the population completed just two generations. The best sequence at the
end of the experiment contained only Resource~1. The estimated fitness of this
sequence is 0.4388. The number of evaluations is however too low to put much
value on the estimate. Figure~\ref{fig:exp_cumul_rules_high} shows the
cumulative regret built-up in \emph{Rules high}. The green line indicating the
regret compared to the best of the population cannot be seen, because it is
identical to the regret compared to the best of the generation. This can be
explained by the fact that the best sequence at the end of the experiment is
already present in the first generation and remains present in each consecutive
generation. In the third generation, \emph{Sequence~[3,1]} enters the population
by means of immigration from the \emph{Rules low} population. The new sequence
is tried four times in a row and two times later on, but the measured fitness
was lower than the current best sequence.

\begin{figure}[ht]
	\begin{subfigure}{0.9\linewidth}
	\centering
	\includegraphics[width=\linewidth]{images/results/plot_exp_eval_rules_low_372_31.png}
	\caption{Sequence~[3,1] in the \emph{Rules low} population}
	\label{fig:exp_eval_rules_low_372}
	\end{subfigure}
	\hfill
	\begin{subfigure}{0.9\linewidth}
	\centering
	\includegraphics[width=\linewidth]{images/results/plot_exp_eval_rules_high_1_1.png}
	\caption{Sequence~[1] in the \emph{Rules high} population}
	\label{fig:exp_eval_rules_high_1}
	\end{subfigure}
	\caption[Evaluations of best sequences in Rules]{Observed normalized learning gain values (blue) plotted with the
		running average (green) of the best sequences for the \emph{Rules}
	lesson.}
	\label{fig:exp_eval_rules}
\end{figure}

\section{Lesson: Intuition}
\label{sec:results_intuition}
Figure~\ref{fig:exp_cumul_intuition_low} shows the cumulative regret of the
\emph{Intuition low} population. During the experiment 108 evaluations were
received from students. The best sequence at the end of the generation is
\emph{Sequence~[6]}. The sequence was already present in the first generation
and remained present throughout all eleven generations. Because of that the
cumulative regret compared to the best of the generation is equal to that
compared to the best of the population. The estimated fitness of
\emph{Sequence~[6]} is 0.4576. The range of observed fitness values for
\emph{Sequence~[6]} is plotted in Figure~\ref{fig:exp_eval_intuition_low_14}.
\\\\
\noindent
In the second generation of the \emph{Intuition low} population, the
\emph{TutOER} system tried out the sequence of Resource~5 and Resource~6
several times in both orderings. Both however scored lower than the best
sequence in that generation. As a result, in the third generation almost all
individuals contained \emph{Sequence~[6]}. This is also visualized in
Table~\ref{tab:results_diversity} that shows the diversity of the
\emph{Intuition low} population dropped to 29\%, which means there are two
types of sequences in the generation. From the second generation onwards, the
\emph{TutOER} system sticks to presenting \emph{Sequence~6}. The only exception
is the 61st evaluation where \emph{Sequence~[6,8]}, coming from the
\emph{Intuition high} population, is tried once. At the end of the experiment
seven out of the 40 sequences were encountered, which is also shown in
Figure~\ref{fig:exp_cover_intuition_low}.\\\\
\noindent
The best sequence in the \emph{Intuition high}
population at the end of the experiment is \emph{Sequence~[6,8]}. The estimated
fitness value of this sequence is 0.5959. However, the \emph{TutOER} system did
not manage to hold on to this sequence. As a result, it was only evaluated ten
times. Figure~\ref{fig:exp_eval_intuition_high_373} displays the observed
fitness values .After the first generation, three consecutive generations were
dominated by \emph{Sequence~[6]}. All other sequences that were present in the
first generation had a negative estimated fitness. In the roulette wheel
selection these sequences have a zero chance of being selected. Furthermore,
the elite member slots were filled with the two occurences of
\emph{Sequence~[6]}. Table~\ref{tab:results_diversity} shows the drop in
diversity as a result. Figure~\ref{fig:exp_cover_intuition_high} shows that at
the end of the experiment six out of the 40 sequences were encountered.\\\\
\noindent
In the fifth generation a mutation on \emph{Sequence~[6]} resulted in
\emph{Sequence~[6,8]}. The \emph{TutOER} system lost the sequence in the
transition to the seventh generation. This is likely due to a bug in the
software related to sequences that migrated. As a consequence the aggregated
fitness value of those sequences was altered by both populations in which it
was present. This aggregated value was used by both the UCB algorithm and
roulette wheel selection. As a result, \emph{Sequence~[6]} appeared to have a
higher fitness than \emph{Sequence~[6,8]} when this was not the case. The bug
was discovered after the experiment was completed. The fitness estimates used
in this report are not affected by this bug.

\begin{figure}[ht]
	\begin{subfigure}{0.9\linewidth}
	\centering
	\includegraphics[width=\linewidth]{images/results/plot_exp_eval_intuition_low_14_6.png}
	\caption{Sequence~[6] in the \emph{Intuition low} population}
	\label{fig:exp_eval_intuition_low_14}
	\end{subfigure}
	\hfill
	\begin{subfigure}{0.9\linewidth}
	\centering
	\includegraphics[width=\linewidth]{images/results/plot_exp_eval_intuition_high_373_68.png}
	\caption{Sequence~[6,8] in the \emph{Intuition high} population}
	\label{fig:exp_eval_intuition_high_373}
	\end{subfigure}
	\caption[Evaluations of best sequences in Intuition]{Observed normalized learning gain values (blue) plotted with the
		running average (green) of the best sequences for the \emph{Intuition}
	lesson.}
	\label{fig:exp_eval_intuition}
\end{figure}

\begin{figure}[ht]
	\begin{subfigure}{0.49\linewidth}
	\centering
	\includegraphics[width=\linewidth]{images/results/plot_exp_cumul_rules_low.png}
	\caption{Rules low}
	\label{fig:exp_cumul_rules_low}
	\end{subfigure}
	\hfill
	\begin{subfigure}{0.49\linewidth}
	\centering
	\includegraphics[width=\linewidth]{images/results/plot_exp_cumul_rules_high.png}
	\caption{Rules high}
	\label{fig:exp_cumul_rules_high}
	\end{subfigure}
	\begin{subfigure}{0.49\linewidth}
	\centering
	\includegraphics[width=\linewidth]{images/results/plot_exp_cumul_intuition_low.png}
	\caption{Intuition low}
	\label{fig:exp_cumul_intuition_low}
	\end{subfigure}
	\hfill
	\begin{subfigure}{0.49\linewidth}
	\centering
	\includegraphics[width=\linewidth]{images/results/plot_exp_cumul_intuition_high.png}
	\caption{Intuition high}
	\label{fig:exp_cumul_intuition_high}
	\end{subfigure}
	\caption[Cumulative regret in Rules and Intuition]{Cumulative regret of the populations \emph{Rules low}, \emph{Rules
	High}, \emph{Intuition low} and \emph{Intuition High}}
	\label{fig:exp_cumul1}
\end{figure}

\begin{figure}[ht]
	\begin{subfigure}{0.49\linewidth}
	\centering
	\includegraphics[width=\linewidth]{images/results/plot_exp_cover_rules_low.png}
	\caption{Rules low}
	\label{fig:exp_cover_rules_low}
	\end{subfigure}
	\hfill
	\begin{subfigure}{0.49\linewidth}
	\centering
	\includegraphics[width=\linewidth]{images/results/plot_exp_cover_rules_high.png}
	\caption{Rules high}
	\label{fig:exp_cover_rules_high}
	\end{subfigure}
	\begin{subfigure}{0.49\linewidth}
	\centering
	\includegraphics[width=\linewidth]{images/results/plot_exp_cover_intuition_low.png}
	\caption{Intuition low}
	\label{fig:exp_cover_intuition_low}
	\end{subfigure}
	\hfill
	\begin{subfigure}{0.49\linewidth}
	\centering
	\includegraphics[width=\linewidth]{images/results/plot_exp_cover_intuition_high.png}
	\caption{Intuition high}
	\label{fig:exp_cover_intuition_high}
	\end{subfigure}
	\caption[Percentage of sequences evaluated in Rules and Intuition]{Percentage of sequences evaluated after each evaluation in \emph{Rules
	low}, \emph{Rules high}, \emph{Intuition low} and \emph{Intuition high}.}
	\label{fig:exp_cover1}
\end{figure}

\begin{figure}[ht]
	\begin{subfigure}{0.49\linewidth}
	\centering
	\includegraphics[width=\linewidth]{images/results/plot_exp_chromosome_ages_rules_low.png}
	\caption{Rules low}
	\label{fig:exp_age_rules_low}
	\end{subfigure}
	\hfill
	\begin{subfigure}{0.49\linewidth}
	\centering
	\includegraphics[width=\linewidth]{images/results/plot_exp_chromosome_ages_rules_high.png}
	\caption{Rules high}
	\label{fig:exp_age_rules_high}
	\end{subfigure}
	\begin{subfigure}{0.49\linewidth}
	\centering
	\includegraphics[width=\linewidth]{images/results/plot_exp_chromosome_ages_intuition_low.png}
	\caption{Intuition low}
	\label{fig:exp_age_intuition_low}
	\end{subfigure}
	\hfill
	\begin{subfigure}{0.49\linewidth}
	\centering
	\includegraphics[width=\linewidth]{images/results/plot_exp_chromosome_ages_intuition_high.png}
	\caption{Intuition high}
	\label{fig:exp_age_intuition_high}
	\end{subfigure}
	\caption[Sorted number of evaluations in Rules and Intuition]{Sorted number of evaluations of each sequence in \emph{Rules
	low}, \emph{Rules high}, \emph{Intuition low} and \emph{Intuition high}.}
	\label{fig:exp_ages1}
\end{figure}

\section{Lesson: Binary}
\label{sec:results_binary}
The \emph{Binary low} population received 132 evaluations by students.
\emph{Sequence~[9]} turned out to be the best with an estimated fitness of
0.5251. Figure~\ref{fig:exp_eval_binary_low_17} shows the observed fitness
values for each of the 103 students that evaluated \emph{Sequence~[9]}.
Figure~\ref{fig:exp_age_binary_low} shows that these 103 evaluations make
\emph{Sequence~[9]} the most evaluated sequence in this population by far. Yet
eleven out of 40 possible sequences have been encountered, as can be seen in
Figure~\ref{fig:exp_cover_binary_low}. This is the highest coverage of the
eight populations.\\\\
\noindent
Figure~\ref{fig:exp_cumul_binary_low} shows the cumulative regret in the
\emph{Binary low} population. The best sequence, \emph{Sequence~[9]}, is
already present in the first generation. Therefore the cumulative regret
compared to the best of the population is identical to the cumulative regret compared
to the best of the generation. From the 40th evaluation, the \emph{TutOER}
system sticks to \emph{Sequence~[9]}. In the seventh generation a few
evaluations were assigned to \emph{Sequence~[9,11]}. However, with an estimated
fitness of 0.4676 this sequence did not outperform \emph{Sequence~[9]}.\\\\
\noindent
Contrary to its counterpart, population \emph{Binary high} has the lowest
coverage of all populations. Figure~\ref{fig:exp_cover_binary_high} shows that
only 4 out of 40 sequences have been encountered. That is the coverage
that every population starts with. No crossover operations were applied in the
transition to the second population. The order of the sampling of survivors
resulted in two pairs of equal chromosomes and one different chromosome at the
end. The parent selection selects pairs from this sampled list of survivors
from top to bottom to be potential parents. The first two pairs of
parents were identical and therefore unfit for crossover. The last survivor
could not be paired with anything. As a result, the second generation of the
\emph{Binary high} population was filled with survivors and not with their
offspring.\\\\
\noindent
The third generation contained only \emph{Sequence~[12]} apart from the two
elite members that carried \emph{Sequence~[9]}. When only the same sequence is
sampled, no crossover can be applied. The reasoning is similar to that in the
previous paragraph. Table~\ref{tab:results_diversity} shows the reduction in
diversity in three generations.\\\\
\noindent
After 27 evaluations, the best sequence in \emph{Binary high} is
\emph{Sequence~[10]} with an estimated fitness of 0.3265. However,
Figure~\ref{fig:exp_cumul_binary_high} shows the \emph{TutOER} system does
not hold on to the best sequence. The bug relating to immigrated chromosomes
that was discussed in Section~\ref{sec:results_intuition} is likely the cause
of that. \emph{Sequence~[9]} that immigrated from \emph{Binary low} appeared to
be the best performing sequence. Even though \emph{Sequence~[10]} was in fact
the only sequence with a non-negative fitness estimate. However, with so few
evaluations it is hard to say whether \emph{Sequence~[10]} was actually good.

\begin{figure}[ht]
	\begin{subfigure}{0.9\linewidth}
	\centering
	\includegraphics[width=\linewidth]{images/results/plot_exp_eval_binary_low_17_9.png}
	\caption{Sequence~[9] in the \emph{Binary low} population}
	\label{fig:exp_eval_binary_low_17}
	\end{subfigure}
	\hfill
	\begin{subfigure}{0.9\linewidth}
	\centering
	\includegraphics[width=\linewidth]{images/results/plot_exp_eval_binary_high_22_10.png}
	\caption{Sequence~[10] in the \emph{Binary high} population}
	\label{fig:exp_eval_binary_high_10}
	\end{subfigure}
	\caption[Evaluations of the best sequences in Binary]{Observed normalized learning gain values (blue) plotted with the
		running average (green) of the best sequences for the \emph{Binary}
	lesson.}
	\label{fig:exp_eval_binary}
\end{figure}

\section{Lesson: Nim-sum}
\label{sec:results_nimsum}
The \emph{Nim-sum low} population gathered 166 evaluations by students at the
end of the experiment. From the 15th evaluation onwards, the \emph{TutOER}
system mainly stuck to evaluating \emph{Sequence~[16]}. This is also reflected
in the cumulative regret curves shown in Figure~\ref{fig:exp_cumul_nimsum_low}.
With an estimated fitness of 0.2557, \emph{Sequence~[16]} turned out to be the
best sequence of the population. Figure~\ref{fig:exp_age_nimsum_low} shows the
difference between the number of evaluations of each sequence. The difference
between the first and second most evaluated sequence is 150 evaluations. This
difference is the largest of all other populations, but \emph{Rules low} comes
close with a difference of 148 evaluations. Also similar to in \emph{Rules
low}, eight out of 40 possible sequences were encountered. However, the
diversity in each generation is rather low. From the fourth to the eighth
generation, only one sequence was present. Table~\ref{tab:results_diversity}
shows the diversity per generation.\\\\
\noindent
The least matured population is \emph{Nim-sum high} with only 22 evaluations.
Despite the limited number of evaluations, ten out of 40 sequences were
encountered (Figure~\ref{fig:exp_cover_nimsum_high}).
Table~\ref{tab:results_diversity} shows the high diversity that is the cause of
this. The best sequence is \emph{Sequence~[15,16]} with an estimated fitness of
1.0. As a result, the cumulative regret compared to the best of the population
and the cumulative regret compared to the theoretical best is identical in
Figure~\ref{fig:exp_cumul_nimsum_high}. However, this sequence has only been
evaluated once. The variance in the fitness of all other learning materials
shows that one evaluation provides very little information.

\begin{figure}[ht]
	\begin{subfigure}{0.9\linewidth}
	\centering
	\includegraphics[width=\linewidth]{images/results/plot_exp_eval_nimsum_low_28_16.png}
	\caption{Sequence~[16] in the \emph{Nim-sum low} population}
	\label{fig:exp_eval_nimsum_low_28}
	\end{subfigure}
	\caption[Evaluations of the best sequences in Nim-sum]{Observed normalized learning gain values (blue) plotted with the
		running average (green) of the best sequences for the \emph{Nim-Sum}
	lesson.}
	\label{fig:exp_eval_nimsum}
\end{figure}

\begin{figure}[ht]
	\begin{subfigure}{0.49\linewidth}
	\centering
	\includegraphics[width=\linewidth]{images/results/plot_exp_cumul_binary_low.png}
	\caption{Binary low}
	\label{fig:exp_cumul_binary_low}
	\end{subfigure}
	\hfill
	\begin{subfigure}{0.49\linewidth}
	\centering
	\includegraphics[width=\linewidth]{images/results/plot_exp_cumul_binary_high.png}
	\caption{Binary high}
	\label{fig:exp_cumul_binary_high}
	\end{subfigure}
	\begin{subfigure}{0.49\linewidth}
	\centering
	\includegraphics[width=\linewidth]{images/results/plot_exp_cumul_nimsum_low.png}
	\caption{Nim-sum low}
	\label{fig:exp_cumul_nimsum_low}
	\end{subfigure}
	\hfill
	\begin{subfigure}{0.49\linewidth}
	\centering
	\includegraphics[width=\linewidth]{images/results/plot_exp_cumul_nimsum_high.png}
	\caption{Nim-sum high}
	\label{fig:exp_cumul_nimsum_high}
	\end{subfigure}
	\caption[Cumulative regret in Binary and Nim-sum]{Cumulative regret of the populations \emph{Binary low},
	\emph{Binary High}, \emph{Nim-sum low} and \emph{Nim-sum High}}
	\label{fig:exp_cumul2}
\end{figure}

\begin{figure}[ht]
	\begin{subfigure}{0.49\linewidth}
	\centering
	\includegraphics[width=\linewidth]{images/results/plot_exp_chromosome_ages_binary_low.png}
	\caption{Binary low}
	\label{fig:exp_age_binary_low}
	\end{subfigure}
	\hfill
	\begin{subfigure}{0.49\linewidth}
	\centering
	\includegraphics[width=\linewidth]{images/results/plot_exp_chromosome_ages_binary_high.png}
	\caption{Binary high}
	\label{fig:exp_age_binary_high}
	\end{subfigure}
	\begin{subfigure}{0.49\linewidth}
	\centering
	\includegraphics[width=\linewidth]{images/results/plot_exp_chromosome_ages_nimsum_low.png}
	\caption{Nim-sum low}
	\label{fig:exp_age_nimsum_low}
	\end{subfigure}
	\hfill
	\begin{subfigure}{0.49\linewidth}
	\centering
	\includegraphics[width=\linewidth]{images/results/plot_exp_chromosome_ages_nimsum_high.png}
	\caption{Nim-sum high}
	\label{fig:exp_age_nimsum_high}
	\end{subfigure}
	\caption[Sorted number of evaluations in Binary and Nim-sum]{Sorted number of evaluations of each sequence in \emph{Binary
	low}, \emph{Binary high}, \emph{Nim-sum low} and \emph{Nim-sum high}.}
	\label{fig:exp_ages2}
\end{figure}

%\begin{figure}[ht]
%	\centering
%	\includegraphics[width=0.98\linewidth]{images/results/plot_exp_chromosome_ages.png}
%	\caption{Evaluations of all chromosomes, sorted.}
%	\label{fig:exp_ages_all}
%\end{figure}


\begin{figure}[ht]
	\begin{subfigure}{0.49\linewidth}
	\centering
	\includegraphics[width=\linewidth]{images/results/plot_exp_cover_binary_low.png}
	\caption{Binary low}
	\label{fig:exp_cover_binary_low}
	\end{subfigure}
	\hfill
	\begin{subfigure}{0.49\linewidth}
	\centering
	\includegraphics[width=\linewidth]{images/results/plot_exp_cover_binary_high.png}
	\caption{Binary high}
	\label{fig:exp_cover_binary_high}
	\end{subfigure}
	\begin{subfigure}{0.49\linewidth}
	\centering
	\includegraphics[width=\linewidth]{images/results/plot_exp_cover_nimsum_low.png}
	\caption{Nim-sum low}
	\label{fig:exp_cover_nimsum_low}
	\end{subfigure}
	\hfill
	\begin{subfigure}{0.49\linewidth}
	\centering
	\includegraphics[width=\linewidth]{images/results/plot_exp_cover_nimsum_high.png}
	\caption{Nim-sum high}
	\label{fig:exp_cover_nimsum_high}
	\end{subfigure}
	\caption[Percentage sequences evaluated in Binary and Nim-sum]{Percentage of sequences evaluated after each evaluation in
	\emph{Binary low}, \emph{Binary high}, \emph{Nim-sum low} and \emph{Nim-sum
	high}.}
	\label{fig:exp_cover2}
\end{figure}
