%%%%%%%%%%%%%%%%%%%%
% Chapter: Results %
%%%%%%%%%%%%%%%%%%%%
In this chapter the result of the experiment are enumerated and discussed.
Although the results will be analyzed per population, most of the
visualizations are grouped together. These visualizations will first be
explainaed by Section~\ref{sec:results_explanation}.
Section~\ref{sec:results_general} then lists some general statistics of the
experiment.
Sections~\ref{sec:results_rules}~through~\ref{sec:results_nimsum}
analyse the results per lesson.

\section{Explanation of the visualizations}
\label{sec:results_explanation}
\subsection{Three curves of cumulative regret}
Figure~\ref{fig:exp_cumul1} and Figure~\ref{fig:exp_cumul2} show the cumulative
regret curves per population. Each subfigure contains three curves, which are
labelled \emph{theoretical best}, \emph{population best} and \emph{generation
best}. The curve labelled \emph{theoretical best} shows the cumulative regret
compared to the perfect score of a normalized learning gain of 1. This score
could only be achieved if all evaluations of a particular sequence returned the
maximum result of 1, something which is not likely in a noisy environment. That
is why two more curves have been plotted. The curve labelled \emph{population
best} plots the cumulative regret compared to the estimated fitness of the best
sequence ever evaluated by the \emph{TutOER} system in that population.
Similarly, the \emph{generation best} labelled curve shows the cumulative
regret compared to the estimated fitness of the best sequence available in the
current generation. For all fitness values, the estimated value at the end of
the experiment was used while calculating the regret. The reason for that is
that comparing with the estimated values at the moment of the decision is less
interesting. Theoretical results from the literature about genetic
algorithm and UCB already tell us how they will deal with exploration versus
exploitation and the regret that would be caused by that. Furthermore from the
simulations performed in this thesis it appears that the algorithm is properly
implemented. What the experiment is intended to find out is whether the task of
OER quality assessment by curriculum sequencing is feasible in a realistic
scenario with noise due to real students. For that purpose it is more
interesting to see whether the \emph{TutOER} system would temporarily discard
optimal sequences that have seem to perform poorly at the time, or whether the
system is robust enough.
\subsection{Fitness variance}
\subsection{Sorted number of evaluations}
\subsection{Coverage curve}
\subsection{Diversity table}
Table~\ref{tab:results_diversity} shows the percentage of unique sequences at
each generation for each population. A diversity of 40\% indicates that across
the ten individuals four distinct chromosomes were divided. Similarly, a
diversity of 10\% means that all individuals have the same chromosome.
% TODO : DEAL WITH BUG IN CODE THAT HAS SOME CHROMOSOMES DOUBLE?

\section{General statistics}
\label{sec:results_general}
\section{Lesson: Rules}
\label{sec:results_rules}
The \emph{Rules low} population completed 22 generations and received the
223 evaluations by students. At the end of the experiment
this population converged to the best sequence encounter till that point. The
sequence contains Resource~3 and Resource~1, in that order, and will be
referred to as \emph{Sequence~372} for the rest of the chapter. A graph of all
evaluations of this sequence can be seen in
Figure~\ref{fig:exp_eval_rules_low_372}.
The estimated fitness of \emph{Sequence~372} at the end of the experiment was
0.7634. \emph{Sequence~372} was created by a mutation on Resource~3 in the transition
from the fourth to the fifth generation that added Resource~1. At first,
Resource~1 dissapeared from the population after not being selected for the
second generation. Resource~3 had the highest fitness of the first four
generations \\\\
\noindent
Figure~\ref{fig:exp_cumul_rules_low} shows that after twenty evaluations the
sequences selected by the \emph{TutOER} system turn out to be the best of each
generation. After 40 evaluations, in the fifth generation, \emph{TutOER}
switches to the evaluation of \emph{Sequence~372}. This is only deviated from at
two points. First, at the 81st evaluation, the \emph{TutOER} system tried out
Resource~1 once more. Second, from evaluation 141 till 144 the \emph{TutOER}
system evaluated the mirrored version of \emph{Sequence~372} that ended with
Resource~3. In both cases however, the \emph{TutOER} system returned to the
optimal sequence.\\\\
\noindent
Figure~\ref{fig:exp_cover_rules_low} shows the \emph{Rules low} population
encountered a fifth of all sequences, namely eight out of 40. Although this was
enough to find \emph{Sequence~372}, the large majority of sequences was not
attempted. No sequences of three resources were attempted. These sequences
filled 60\% of the total number of possible sequences in this experiment. When
comparing to the possible sequences of two resources maximum, the \emph{TutOER}
system evaluated half.\\\\
\noindent
The \emph{Rules high} population encountered five out of 40 possible sequences.
However, in contrast to \emph{Rules low}, only 30 evaluations were collected
As a result, the population completed just two generations. The best sequence at the
end of the experiment contained only Resource~1. The estimated fitness of this
sequence is 0.4388. The number of evaluations is however too low to put much
value on the estimate.\\\\
\noindent

% immigrant story:
% Due to a bug in the system, immigrants were not disconnected from their
% original source. As a result, the fitness of both chromosomes were connected
% Two sequences with Gene 1 existed in Rules high.
% Sequence 31 came into the third generation from Rules Low by means of immigration
\section{Lesson: Intuition}
\label{sec:results_intuition}
% Intuition low did not converge, intuition high did converge
% Still lot of noise in Intuition high
% intuition low had 7/40, high, 6/40
% intuition high dropped in diversity dramatically after the first generation,
% because all sequences other than [6]
\section{Lesson: Binary}
\label{sec:results_binary}
\section{Lesson: Nim-sum}
\label{sec:results_nimsum}

\begin{table}
	\begin{subtable}{\linewidth}
	\centering
		\begin{tabular}{lllllllllllll}\hline
		Population & 1 & 2 & 3 & 4 & 5 & 6 & 7 & 8 & 9 & 10 & 11 & 12\\\hline
		Rules low & 40 & 40 & 30 & 40 & 30 & 20 & 10 & 10 & 20 & 10 & 20 & 20\\
		Rules high & 40 & 20 & 30 & \textemdash & \textemdash & \textemdash & \textemdash & \textemdash & \textemdash & \textemdash & \textemdash & \textemdash\\
		Intuition low & 40 & 30 & 20 & 20 & 20 & 20 & 30 & 10 & 20 & 20 & 30 & \textemdash\\
		Intuition high & 40 & 10 & 10 & 10 & 20 & 20 & 10 & 30 & 30 & 10 & 20 & \textemdash\\
		Binary low & 40 & 40 & 30 & 60 & 20 & 10 & 30 & 30 & 20 & 30 & 30 & 10\\
		Binary high & 40 & 30 & 40 & \textemdash & \textemdash & \textemdash & \textemdash & \textemdash & \textemdash & \textemdash & \textemdash & \textemdash\\
		Nim-sum low & 40 & 40 & 20 & 10 & 10 & 10 & 10 & 10 & 20 & 10 & 10 & 30\\
		Nim-sum high & 40 & 60 & 60 & \textemdash & \textemdash & \textemdash & \textemdash & \textemdash & \textemdash & \textemdash & \textemdash & \textemdash\\
		\end{tabular}
	\caption{generation 1-12}
	\end{subtable}

	\medskip
	\begin{subtable}{\linewidth}
	\centering
		\begin{tabular}{lllllllllllll}\hline
		Population & 13 & 14 & 15 & 16 & 17 & 18 & 19 & 20 & 21 & 22 & 23 & 24\\\hline
		Rules low & 30 & 20 & 20 & 20 & 20 & 20 & 20 & 10 & 10 & 20 & 30 & \textemdash\\
		Rules high & \textemdash & \textemdash & \textemdash & \textemdash & \textemdash & \textemdash & \textemdash & \textemdash & \textemdash & \textemdash & \textemdash & \textemdash\\
		Intuition low & \textemdash & \textemdash & \textemdash & \textemdash & \textemdash & \textemdash & \textemdash & \textemdash & \textemdash & \textemdash & \textemdash & \textemdash\\
		Intuition high & \textemdash & \textemdash & \textemdash & \textemdash & \textemdash & \textemdash & \textemdash & \textemdash & \textemdash & \textemdash & \textemdash & \textemdash\\
		Binary low & 20 & 10 & \textemdash & \textemdash & \textemdash & \textemdash & \textemdash & \textemdash & \textemdash & \textemdash & \textemdash & \textemdash\\
		Binary high & \textemdash & \textemdash & \textemdash & \textemdash & \textemdash & \textemdash & \textemdash & \textemdash & \textemdash & \textemdash & \textemdash & \textemdash\\
		Nim-sum low & 10 & 10 & 10 & 10 & 10 & \textemdash & \textemdash & \textemdash & \textemdash & \textemdash & \textemdash & \textemdash\\
		Nim-sum high & \textemdash & \textemdash & \textemdash & \textemdash & \textemdash & \textemdash & \textemdash & \textemdash & \textemdash & \textemdash & \textemdash & \textemdash\\	
		\end{tabular}
	\caption{generation 13-24}
	\end{subtable}
	\caption{Diversity percentage at each generation}
	\label{tab:results_diversity}
\end{table}

\begin{figure}[ht]
	\begin{subfigure}{0.49\linewidth}
	\centering
	\includegraphics[width=\linewidth]{images/results/plot_exp_cumul_rules_low.png}
	\caption{Rules low}
	\label{fig:exp_cumul_rules_low}
	\end{subfigure}
	\hfill
	\begin{subfigure}{0.49\linewidth}
	\centering
	\includegraphics[width=\linewidth]{images/results/plot_exp_cumul_rules_high.png}
	\caption{Rules high}
	\label{fig:exp_cumul_rules_high}
	\end{subfigure}
	\begin{subfigure}{0.49\linewidth}
	\centering
	\includegraphics[width=\linewidth]{images/results/plot_exp_cumul_intuition_low.png}
	\caption{Intuition low}
	\label{fig:exp_cumul_intuition_low}
	\end{subfigure}
	\hfill
	\begin{subfigure}{0.49\linewidth}
	\centering
	\includegraphics[width=\linewidth]{images/results/plot_exp_cumul_intuition_high.png}
	\caption{Intuition high}
	\label{fig:exp_cumul_intuition_high}
	\end{subfigure}
	\caption{Cumulative regret of the populations \emph{Rules low}, \emph{Rules
	High}, \emph{Intuition low} and \emph{Intuition High}}
	\label{fig:exp_cumul1}
\end{figure}
\begin{figure}[ht]
	\begin{subfigure}{0.49\linewidth}
	\centering
	\includegraphics[width=\linewidth]{images/results/plot_exp_cumul_binary_low.png}
	\caption{Binary low}
	\label{fig:exp_cumul_binary_low}
	\end{subfigure}
	\hfill
	\begin{subfigure}{0.49\linewidth}
	\centering
	\includegraphics[width=\linewidth]{images/results/plot_exp_cumul_binary_high.png}
	\caption{Binary high}
	\label{fig:exp_cumul_binary_high}
	\end{subfigure}
	\begin{subfigure}{0.49\linewidth}
	\centering
	\includegraphics[width=\linewidth]{images/results/plot_exp_cumul_nimsum_low.png}
	\caption{Nim-sum low}
	\label{fig:exp_cumul_nimsum_low}
	\end{subfigure}
	\hfill
	\begin{subfigure}{0.49\linewidth}
	\centering
	\includegraphics[width=\linewidth]{images/results/plot_exp_cumul_nimsum_high.png}
	\caption{Nim-sum high}
	\label{fig:exp_cumul_nimsum_high}
	\end{subfigure}
	\caption{Cumulative regret of the populations \emph{Binary low},
	\emph{Binary High}, \emph{Nim-sum low} and \emph{Nim-sum High}}
	\label{fig:exp_cumul2}
\end{figure}


\begin{figure}[ht]
	\begin{subfigure}{0.9\linewidth}
	\centering
	\includegraphics[width=\linewidth]{images/results/plot_exp_eval_rules_low_372_31.png}
	\caption{Sequence~372 in the \emph{Rules low} population}
	\label{fig:exp_eval_rules_low_372}
	\end{subfigure}
	\hfill
	\begin{subfigure}{0.9\linewidth}
	\centering
	\includegraphics[width=\linewidth]{images/results/plot_exp_eval_intuition_high_14_6.png}
	\caption{Sequence~14 in the \emph{Intuition high} population}
	\label{fig:exp_eval_innovation_high_14}
	\end{subfigure}
	\caption{Observed normalized learning gain values (blue) plotted with the
	running average (green).}
	\label{fig:exp_eval}
\end{figure}

\begin{figure}[ht]
	\begin{subfigure}{0.49\linewidth}
	\centering
	\includegraphics[width=\linewidth]{images/results/plot_exp_chromosome_ages_rules_low.png}
	\caption{Rules low}
	\label{fig:exp_age_rules_low}
	\end{subfigure}
	\hfill
	\begin{subfigure}{0.49\linewidth}
	\centering
	\includegraphics[width=\linewidth]{images/results/plot_exp_chromosome_ages_rules_high.png}
	\caption{Rules high}
	\label{fig:exp_age_rules_high}
	\end{subfigure}
	\begin{subfigure}{0.49\linewidth}
	\centering
	\includegraphics[width=\linewidth]{images/results/plot_exp_chromosome_ages_intuition_low.png}
	\caption{Intuition low}
	\label{fig:exp_age_intuition_low}
	\end{subfigure}
	\hfill
	\begin{subfigure}{0.49\linewidth}
	\centering
	\includegraphics[width=\linewidth]{images/results/plot_exp_chromosome_ages_intuition_high.png}
	\caption{Intuition high}
	\label{fig:exp_age_intuition_high}
	\end{subfigure}
	\caption{Sorted number of evaluations of each sequence in \emph{Rules
	low}, \emph{Rules high}, \emph{Intuition low} and \emph{Intuition high}.}
	\label{fig:exp_ages1}
\end{figure}

\begin{figure}[ht]
	\begin{subfigure}{0.49\linewidth}
	\centering
	\includegraphics[width=\linewidth]{images/results/plot_exp_chromosome_ages_binary_low.png}
	\caption{Binary low}
	\label{fig:exp_age_binary_low}
	\end{subfigure}
	\hfill
	\begin{subfigure}{0.49\linewidth}
	\centering
	\includegraphics[width=\linewidth]{images/results/plot_exp_chromosome_ages_binary_high.png}
	\caption{Binary high}
	\label{fig:exp_age_binary_high}
	\end{subfigure}
	\begin{subfigure}{0.49\linewidth}
	\centering
	\includegraphics[width=\linewidth]{images/results/plot_exp_chromosome_ages_nimsum_low.png}
	\caption{Nim-sum low}
	\label{fig:exp_age_nimsum_low}
	\end{subfigure}
	\hfill
	\begin{subfigure}{0.49\linewidth}
	\centering
	\includegraphics[width=\linewidth]{images/results/plot_exp_chromosome_ages_nimsum_high.png}
	\caption{Nim-sum high}
	\label{fig:exp_age_nimsum_high}
	\end{subfigure}
	\caption{Sorted number of evaluations of each sequence in \emph{Binary
	low}, \emph{Binary high}, \emph{Nim-sum low} and \emph{Nim-sum high}.}
	\label{fig:exp_ages2}
\end{figure}

%\begin{figure}[ht]
%	\centering
%	\includegraphics[width=0.98\linewidth]{images/results/plot_exp_chromosome_ages.png}
%	\caption{Evaluations of all chromosomes, sorted.}
%	\label{fig:exp_ages_all}
%\end{figure}

\begin{figure}[ht]
	\begin{subfigure}{0.49\linewidth}
	\centering
	\includegraphics[width=\linewidth]{images/results/plot_exp_cover_rules_low.png}
	\caption{Rules low}
	\label{fig:exp_cover_rules_low}
	\end{subfigure}
	\hfill
	\begin{subfigure}{0.49\linewidth}
	\centering
	\includegraphics[width=\linewidth]{images/results/plot_exp_cover_rules_high.png}
	\caption{Rules high}
	\label{fig:exp_cover_rules_high}
	\end{subfigure}
	\begin{subfigure}{0.49\linewidth}
	\centering
	\includegraphics[width=\linewidth]{images/results/plot_exp_cover_intuition_low.png}
	\caption{Intuition low}
	\label{fig:exp_cover_intuition_low}
	\end{subfigure}
	\hfill
	\begin{subfigure}{0.49\linewidth}
	\centering
	\includegraphics[width=\linewidth]{images/results/plot_exp_cover_intuition_high.png}
	\caption{Intuition high}
	\label{fig:exp_cover_intuition_high}
	\end{subfigure}
	\caption{Percentage of sequences evaluated after each evaluation in \emph{Rules
	low}, \emph{Rules high}, \emph{Intuition low} and \emph{Intuition high}.}
	\label{fig:exp_cover1}
\end{figure}

\begin{figure}[ht]
	\begin{subfigure}{0.49\linewidth}
	\centering
	\includegraphics[width=\linewidth]{images/results/plot_exp_cover_binary_low.png}
	\caption{Binary low}
	\label{fig:exp_cover_binary_low}
	\end{subfigure}
	\hfill
	\begin{subfigure}{0.49\linewidth}
	\centering
	\includegraphics[width=\linewidth]{images/results/plot_exp_cover_binary_high.png}
	\caption{Binary high}
	\label{fig:exp_cover_binary_high}
	\end{subfigure}
	\begin{subfigure}{0.49\linewidth}
	\centering
	\includegraphics[width=\linewidth]{images/results/plot_exp_cover_nimsum_low.png}
	\caption{Nim-sum low}
	\label{fig:exp_cover_nimsum_low}
	\end{subfigure}
	\hfill
	\begin{subfigure}{0.49\linewidth}
	\centering
	\includegraphics[width=\linewidth]{images/results/plot_exp_cover_nimsum_high.png}
	\caption{Nim-sum high}
	\label{fig:exp_cover_nimsum_high}
	\end{subfigure}
	\caption{Percentage of sequences evaluated after each evaluation in
	\emph{Binary low}, \emph{Binary high}, \emph{Nim-sum low} and \emph{Nim-sum
	high}.}
	\label{fig:exp_cover2}
\end{figure}
