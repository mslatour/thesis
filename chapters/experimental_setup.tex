%%%%%%%%%%%%%%%%%%%%%%%%%%%%%%%
% Chapter: Experimental setup %
%%%%%%%%%%%%%%%%%%%%%%%%%%%%%%%
\section{GA Parameters}
\begin{description}
	\item[Number of episodes] 4
	\item[Number of individuals in population] 4
	\item[Number of elite individuals] 2
	\item[Probability of mutation] 0.05
	\item[Minimum length of chromosome] 1
	\item[Maximum length of chromosome] 3
\end{description}
\section{Student groups}
In order to reduce the number of required data points, the students were split
up in only two different student groups per knowledge component. The splitting
criteria was based on the achieved pre-test score. If a student scored more
than 50\% then he or she was assigned to the High group of that knowledge
component, else he or she was assigned to the Low group of that knowledge
component. This resulted in the following groups.\\\\
\begin{tabular}{l|l||l}
	\textbf{Knowledge Component} & \textbf{Pre-test score}    & \textbf{Group} \\\hline\hline
	Rules of the game	& $\leq 50\%$		& Rules Low \\\hline
	Rules of the game	& $> 50\%$			& Rules High \\\hline
	Intuition			& $\leq 50\%$		& Intuition Low \\\hline
	Intuition			& $> 50\%$			& Intuition High \\\hline
	Binary Numbers		& $\leq 50\%$		& Binary Low \\\hline
	Binary Numbers		& $> 50\%$			& Binary High \\\hline
	Nim-Sum				& $\leq 50\%$		& NimSum Low \\\hline
	Nim-Sum				& $> 50\%$			& NimSum High \\\hline
\end{tabular}
\section{Educational material}
% TODO: Describe initial page of curriculum with explanation of the experiment
Participants in the experiment went through a curriculum about the game of Nim.
Nim is a mathematical game where two players alternate turns in taking away at
least one object from exactly one of stacks of objects on the table. The
experiment only looks at the normal version of the Nim game, where the person
that takes the last object of the table wins. In the mis\`{e}re version of the
game, this person would have lost. Nim is a zero-sum game and is part of a
large collection of related games which have mathematically grounded
strategies. Nim's winning strategy, provided you are in a winnable position,
involves applying the nim-sum operator. % Explain the nim-sum operator

The Nim curriculum is divided into four knowledge components. First, the rules
of the game are explained. Second, some intuition about good strategies is formed.
Third, binary numbers are covered. Fourth, the binary nim-sum operation is
explained. For each of these knowledge component, four educational resources
were created from scratch or composed of explanations found on the internet.
For the genetic algorithm to have a chance of learning something, it is
important that the collection of resources contains both good and bad material.
The resources for each knowledge component are described in
Appendix~\ref{ax_resources}, the rest of this section will describe
each knowledge component.
\subsection{Rules of the game}
\subsection{Intuition}
\subsection{Binary Numbers}
\subsection{Nim-Sum}
% explain nim-sum with pair cancelling, explain that there is also an XOR
% method to this and that pair cancelling is the more human friendly way
\section{Pre-test \& post-test}
In this experiment, each knowledege component has the same questions for both
the pre-test and post-test. Given the low threshold for participants to stop
participating in the experiment it was necessary to keep the time needed to
fill in the questions limited. This resulted in three multiple-choice
questions for each knowledge component. All questions have at least the option that
indicates that the user doesn't know the answer. This is still counted as a
wrong answer, but is more informative for later analysis than a random guess
which could be right by accident. The questions for each knowledge component
are listed in Appendix~\ref{ax_questions}.

\section{Exam}
\begin{figure}[ht]
    \centering
    \includegraphics[width=0.9\linewidth]{images/screen_exam.png}
    \caption{Screenshot of the exam setup with five nim game scenarios}
    \label{fig:screen_exam}
\end{figure}
After having completed all knowledge components, a student is presented a final
exam. The purpose of this exam is to have an opportunity to compare students at
the beginning of the experiment with those at the end. The exam questions are
nim game scenarios that the student has to play. The game interface is
described in Section~\ref{sec:apx_nimjs}. If the student wins, the
question is answered correctly. If the computer wins, the question is counted
as wrong. The exam grade is calculated as the percentage of won games. The result
of each game is displayed in a progress bar, as can be seen in
Figure~\ref{fig:screen_exam}. The exam is introduced with the following explanation.
\begin{framed}\noindent
	You will be playing 5 nim games of varying difficulty. This is your
	opportunity to show your learned skills. The orange rectangles represent
	stacks of objects. The number of objects on the stack is shown by the white
	number in the stack. You can take objects off a stack by clicking on it.
	You will then be asked how many objects you want to take. After you have
	made a move, your artificial counter player will make one as well. The game
	ends when either of you won.
	After you have played all 5 nim games you are finished.
\end{framed}
\noindent The nim game scenarios are not randomly generated and are increasing in
difficulty. The following list enumerates the game configurations for each
question.
\begin{enumerate}
	\item Two stacks consisting of one and two objects.
	\item Three stacks with two, three and two objects.
	\item Three stacks containing four, five and six objects.
	\item Four stacks consisting of one, three, four and five objects.
	\item Four stacks with ten, four, six and nine objects.
\end{enumerate}
\section{Questionaire}
At the end of the course, after all knowledge components and the exam have been
completed by the student. A small optional questionaire is presented to the
student. The questionaire consisted of three questions.
\begin{enumerate}
	\item How good were you at Nim before these lessons?
		\begin{itemize}
			\item I could not play it at all.
			\item I could play it a little bit.
			\item I could play it very well.
		\end{itemize}
	\item How good do you think you are after these lessons?
		\begin{itemize}
			\item I cannot play it at all.
			\item I can play it a little bit.
			\item I can play it very well.
		\end{itemize}
	\item Do you have any last comments you whish to make?
\end{enumerate}
\section{Amazon Mechanical Turk}
