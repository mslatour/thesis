%%%%%%%%%%%%%%%%%%%%%%%%%%%%%%%
% Chapter: Experimental setup %
%%%%%%%%%%%%%%%%%%%%%%%%%%%%%%%
This chapter describes the experiment that has been performed to test the
\emph{TutOER} system. The purpose of the experiment was to verify whether the
presented apporach would also work in a more realistic environment. The
experiment took the form of an online course. In this course, sequences of OER
are presented to a student in four different lessons. The sequences are
selected by the \emph{TutOER} system. Students are requested to answer a few
multiple-choice questions at the beginning and the end of each sequence. These
questions assess the competency level of the student on the topic of the
lesson.\\\\
\noindent
Section~\ref{sec:setup_params} enumerates the parameter values chosen for the
genetic algorithm during this experiment. Section~\ref{sec:setup_groups} lists the
groups to which the students will be assigned. The educational materials that
were used in the course are described in Section~\ref{sec:setup_material}. The
pre-test and post-test questions are listed in Section~\ref{sec:setup_tests}.
Section~\ref{sec:setup_exam} describes the exam that is presented at the end of
the course. The experiment is ended with a questionaire that is presented in
Section~\ref{sec:setup_questionaire}. Section~\ref{sec:setup_dissemination} discusses
how the participants were found.
\section{Experiment}
\subsection{Open Educational Resources}
\label{sec:setup_material}
% TODO: Describe initial page of curriculum with explanation of the experiment
Participants in the experiment went through a curriculum about the game of Nim.
Nim is a mathematical game where two players alternate turns in taking away at
least one object from exactly one of stacks of objects on the table. The
experiment only looks at the normal version of the Nim game, where the person
that takes the last object of the table wins. In the mis\`{e}re version of the
game, this person would have lost. Nim is a zero-sum game and is part of a
large collection of related games which have mathematically grounded
strategies. Nim's winning strategy, provided you are in a winnable position,
involves applying the nim-sum operator. % Explain the nim-sum operator

The Nim curriculum is divided into four knowledge components. First, the rules
of the game are explained. Second, some intuition about good strategies is formed.
Third, binary numbers are covered. Fourth, the binary nim-sum operation is
explained. For each of these knowledge component, four educational resources
were created from scratch or composed of explanations found on the internet.
For the genetic algorithm to have a chance of learning something, it is
important that the collection of resources contains both good and bad material.
The resources for each knowledge component are described in
Appendix~\ref{ax_resources}, the rest of this section will describe
each knowledge component.
\paragraph{Rules of the game} The first knowledge component covers the rules
of the Nim game. The playing field of the game contains two or more stacks of
objects. Nim is played by two players that in turn take away at least
one object from exactly one stack of objects. The player that takes a way the
last object from the playing field wins.
\paragraph{Intuition} The intuition behind a winning strategy is the subject
of the second knowledge component. In any Nim game, it is desirable to leave
your opponent with two equally sized stacks. From that position, you can mimic
any move your opponent made and return to the same situation until only one
object is left for you to take away and win. The understanding that it is
beneficial to leave your opponent with two equal stacks is the intuition behind
the winning strategy for the Nim game.
\paragraph{Binary Numbers} The third knowledge component covers binary numbers.
In order to generalize the intuition to a winning strategy
for any winneable situation a student needs to be able to work with binary
representations of numbers. Specifically, a student needs to be to able to
convert decimal numbers into binary numbers.
\paragraph{Nim-Sum}
The key to winning the Nim game in any winnable situation is to leave the game
in a state where the nim-sum of the stacks is zero. The nim-sum operator is in
fact the \emph{exclusive or} (XOR) operator, which describes the sum of two
binary numbers neglecting all carries between digits. A different way to put
this is that the XOR operator outputs a `one' when exactly one of the two
inputs contains a `one', and 'zero' in any other case. In a winnable situation,
the nim-sum is not equal to zero at the beginning of your turn. You then remove a
number of objects from a single stack, such that the nim-sum becomes zero.
The nim-sum operator can not only be used to determine whether the nim-sum will
be zero after an action, it can also be used to identify which action you can take.

A slightly easier operation to perform mentally is to write down the stacks
underneath each other in binary format, count the number of `ones' in each
digit column ($2^0$, $2^1$, $2^2$ etc) and ensure that each column has an even
number of `ones'. In the resources used in this thesis this method is referred
to as \emph{pair cancelling}.
\subsection{Pre-test \& post-test}
\label{sec:setup_tests}
In this experiment, each knowledege component has the same questions for both
the pre-test and post-test. Given the low threshold for participants to stop
participating in the experiment it was necessary to keep the time needed to
fill in the questions limited. This resulted in three multiple-choice
questions for each knowledge component. All questions have at least the option that
indicates that the user doesn't know the answer. This is still counted as a
wrong answer, but is more informative for later analysis than a random guess
which could be right by accident. The questions for each knowledge component
are listed in Appendix~\ref{ax_questions}.

\subsection{Exam}
\label{sec:setup_exam}
\begin{figure}[ht]
    \centering
    \includegraphics[width=0.9\linewidth]{images/screen_exam.png}
    \caption{Screenshot of the exam setup with five nim game scenarios}
    \label{fig:screen_exam}
\end{figure}
After having completed all knowledge components, a student is presented a final
exam. The purpose of this exam is to have an opportunity to compare students at
the beginning of the experiment with those at the end. The exam questions are
nim game scenarios that the student has to play. The game interface is
described in Section~\ref{sec:apx_nimjs}. If the student wins, the
question is answered correctly. If the computer wins, the question is counted
as wrong. The exam grade is calculated as the percentage of won games. The result
of each game is displayed in a progress bar, as can be seen in
Figure~\ref{fig:screen_exam}. The exam is introduced with the following explanation.
\begin{framed}\noindent
	You will be playing 5 nim games of varying difficulty. This is your
	opportunity to show your learned skills. The orange rectangles represent
	stacks of objects. The number of objects on the stack is shown by the white
	number in the stack. You can take objects off a stack by clicking on it.
	You will then be asked how many objects you want to take. After you have
	made a move, your artificial counter player will make one as well. The game
	ends when either of you won.
	After you have played all 5 nim games you are finished.
\end{framed}
\noindent The nim game scenarios are not randomly generated and are increasing in
difficulty. The following list enumerates the game configurations for each
question.
\begin{enumerate}
	\item Two stacks consisting of one and two objects.
	\item Three stacks with two, three and two objects.
	\item Three stacks containing four, five and six objects.
	\item Four stacks consisting of one, three, four and five objects.
	\item Four stacks with ten, four, six and nine objects.
\end{enumerate}
\subsection{Questionaire}
\label{sec:setup_questionaire}
At the end of the course, after all knowledge components and the exam have been
completed by the student. A small optional questionaire is presented to the
student. The questionaire consisted of two multiple choice questions and one
open ended question.
\begin{enumerate}
	\item How good were you at Nim before these lessons?
		\begin{itemize}
			\item I could not play it at all.
			\item I could play it a little bit.
			\item I could play it very well.
		\end{itemize}
	\item How good do you think you are after these lessons?
		\begin{itemize}
			\item I cannot play it at all.
			\item I can play it a little bit.
			\item I can play it very well.
		\end{itemize}
	\item Do you have any last comments you whish to make?
\end{enumerate}
\subsection{Participants}
\label{sec:setup_dissemination}
\subsubsection{Amazon Mechanical Turk}
Participants for the experiment are partly found via the Amazon Mechanical
Turk\footnote{\url{https://www.mturk.com/mturk/}} service, where tasks that are
currently not possible to execute using artificial intelligence can be done by
humans in return for a small fee in the form of Amazon credits. The site forms
a market place where a \emph{requester} can post a certain \emph{Human
Intelligence Task} (HIT) that he or she wants to have crowdsourced for a
certain price and where \emph{providers} choose which tasks they want to perform.
The \emph{provider} chooses the time and place of the execution, as most if not
all \emph{HIT}s are only bound to existance of an internet connnection. Only
when a \emph{provider} completes the \emph{HIT} to the approval of the
\emph{provider} will he or she be paid the agreed amount. A \emph{provider} is
not forced to finish a \emph{HIT} and can decide to give up participation at
any moment during the \emph{HIT}. Typically a preview is shown of the task
before the \emph{provider} makes the decision to accept.

In the context of this thesis, a \emph{HIT} represents an entire walkthrough of
the experiment including the different knowledge components and the final exam
of one participant. \emph{Providers} were initially paid ten cents for this task,
which was raised to twenty cents during the experiment to speed up the data
collection. All \emph{providers} that completed the experiment were paid,
regardless of the \emph{provider}'s performance or usefulness of the data.
\emph{Providers} that dropped out during the experiment were not paid, but the data
that was collected as a result of their actions so far was still kept.
\subsubsection{Social network}
Other participants were found via Twitter, Facebook, LinkedIn and Reddit. No
further explanation was given about the game or the nature of the experiment.
Participants that arrived at the experiment via these communication lines did
so without a form of financial reward as \emph{providers} did in Mechanical
Turk. The call on social networks was sent out in the second half of the
experiment.

\section{Genetic algorithm setup}
\subsection{Populations}
\label{sec:setup_groups}
Recall that each student group is a separate population. In order to reduce the
number of required data points, the students were split up in only two
different student groups per knowledge component. The splitting criteria was
based on the achieved pre-test score. If a student scored more than 50\% then
he or she was assigned to the High group of that knowledge component, else he
or she was assigned to the Low group of that knowledge component. This resulted
in the following groups.\\\\
\begin{tabular}{lll}\hline
	\textbf{Knowledge Component} & \textbf{Pre-test score}    & \textbf{Group} \\\hline
	Rules of the game	& $\leq 50\%$		& Rules Low \\
	Rules of the game	& $> 50\%$			& Rules High \\
	Intuition			& $\leq 50\%$		& Intuition Low \\
	Intuition			& $> 50\%$			& Intuition High \\
	Binary Numbers		& $\leq 50\%$		& Binary Low \\
	Binary Numbers		& $> 50\%$			& Binary High \\
	Nim-Sum				& $\leq 50\%$		& NimSum Low \\
	Nim-Sum				& $> 50\%$			& NimSum High \\
\end{tabular}
\subsection{Parameters}
\label{sec:setup_params}
Chapter~\ref{ch:simulations} analyzed what the optimal parameter values were
for the genetic algorithm using evaluations based on the handcrafted model.
These parameter values will be used for the experiment as well, apart from one
exception. The number of individuals in the population was set back from ten
(as it was in the \texttt{erik} setup) to seven. This was out of precaution to
avoid the situation where a too large diversity would require more students to
show signs of convergence than would be available to the experiment. This set
the parameter values of the genetic algorithm in the experiment to the
following.\\\\
\begin{tabular}{ll}\hline
	\textbf{Parameter} &  \textbf{Value}\\\hline
	Population size & 7 \\
	Number of episodes & 10 \\
	Number of elite & 2 \\
	Mutation rate & 0.05 \\
	Minimum length & 1 \\
	Maximum length & 3 \\
\end{tabular}

\section{Evaluation}
The purpose of the experiment is to show that the approach taken in this thesis
could work in a more realistic setting with actual students. To determine that,
two types of success indicators are used. Furthermore several other views on
the data are utilized to gain more insight in what happened. That is useful to
find the reason for the success, or lack thereof. Both sets of visualisations
and metrics are described in this section.
\subsection{Indicators of success}
The main question is: does the system learn over time to pick sequences with
more learning impact, instead of picking those who have less impact? This
question is answered using two metrics: cumulative regret and convergence.
\subsubsection{Cumulative regret}
The cumulative regret metric is similar to the one used in evaluating the
simulations. There, regret was defined as the difference in received reward
between the presented sequence and the optimal sequence. However, unlike in the
simulation environment, in the experiment setting the optimal sequence is not
known. We can only compare to the best sequence we have seen so far, which may
or may not be the global optimum. Thus, the cumulative regret is defined as the
cumulative difference between the estimated fitness of the presented sequence
and the highest known estimated fitness, within the same population.\\\\
\noindent
The estimates used in the regret calculation are the estimated values at the
end of the experiment. The reason is that comparing with the estimated values
at the moment of the decision is less interesting. Theoretical results from the
literature about genetic algorithm and UCB already tell us how they will deal
with exploration versus exploitation and the regret that would be caused by
that. Furthermore from the simulations performed in this thesis it appears that
the algorithm is properly implemented. What the experiment is intended to find
out is whether the task of OER quality assessment by curriculum sequencing is
feasible in a realistic scenario with noise due to real students. For that
purpose it is more interesting to see whether the \emph{TutOER} system would
temporarily discard optimal sequences that have seem to perform poorly at the
time, or whether the system is robust enough.\\\\
\noindent
If the approach works, the regret will decrease over time. As a result the
the growth of the cumulative regret will, approximately, stop. The
\emph{TutOER} system will always continue to try out sequences that appear to
be suboptimal, but the intervals between these explorations will increase.
Small peaks in regret are thus expected, but the trend of the line should be flat. The
\textbf{cumulative regret curve} shows the development of the cumulative regret
after each student.\\\\
\noindent
Furthermore, there are quantitative metrics that can be extracted from the
the cumulative regret curve. The final cumulative regret value is not
particularly interesting, since it doesn't capture the trend of the cumulative
regret curve. However, the cumulative regret of the \textbf{first 20\%} and
\textbf{last 20\%} of evaluations do capture the trend, albeit in a low
resolution. If the approach works, the cumulative regret of the last 20\% of
evaluations must be lower than that of the first 20\%. Ideally the last 20\% of
evaluations have no further regret at all, which would result in a cumulative
regret of zero for those evaluations. However, due to the incidental
explorations it is possible that a small peak occurs in the last 20\%
evaluations. In that case, the cumulative regret curve will give a definitive
answer about the trend.
\subsubsection{Convergence point}
In general we want any learning algorithm to converge. The notion of
convergence however needs some adjustment, because of the mentioned incidental
exploration. The used definition of the point of convergence is the evaluation
from which twenty-five consequtive evaluations targeted the same sequence.
This metric does not indicate whether the system converged on presenting the
optimal sequence, only that it converged on some sequence.
\subsection{Insight in behavior}
\subsubsection{Coverage curve}
The coverage is defined as the percentage of sequences that was evaluated at
leats once. This metric is identical to the one used for the simulation
analysis. The total number of possible sequences is 40. That means each new
encountered sequence adds 2.5\% to the coverage. There are limits to what
percentages can theoretically be achieved in each generation. Each population
is initialized with four different sequences in its first generation.
Furthermore, there are seven individuals in each consecutive generation. That
means that the maximum coverage is given by $4+7\cdot(i-1)$, for each $i$th
generation.\\\\
\noindent
More coverage is not necessarily a good thing. The genetic algorithm should
steer the search towards the more promising areas of the search space. However,
the coverage can offer some indication as to how reliable the statements about
optimality are. When a larger area of the search space has been
encountered, it is more likely that the found optimum is a global one.
\subsubsection{Evaluation noise}
The observed learning gain of each sequence is expected to vary per student. The
amount of noise in this signal has an impact on the performance of the system.
The evaluation value for the best sequence of each population is plotted (in
blue) for each student. Alongside, the running average is shown (in green),
which is the estimate of the sequence's fitness after each evaluation.
\subsubsection{Sorted number of evaluations}
The genetic algorithm, together with UCB-1 selection, introduces a bias towards
evaluating certain sequences over others. That is also what they are intended
to do. However, a very skewed distribution of evaluations indicates that
sequences did not receive the same opportunity to prove optimal. Particularly
when the evaluation noise is large.
\subsubsection{Diversity table}
The diversity table shows the percentage of unique sequences at
each generation for each population. A diversity of 43\% indicates that across
the seven individuals three distinct chromosomes were divided. Similarly, a
diversity of 14\% means that all individuals have the same chromosome. The
table does not capture how many occurences each chromosome has. This is also
not relevant. The UCB-1 algorithm selects which sequence to evaluate regardless
of their number of occurences in the generation.
