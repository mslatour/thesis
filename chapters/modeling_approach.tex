\begin{itemize}
  \item refer to \citep{Eiben2007} for listing the important aspects of any evolutionary algorithm that need to be modeled: representation, fitness, population, parent selection, variation operators, survivor selection, initialization and termination conditions
\end{itemize}
\section{Encoding}
\subsection{The resource gene}
\begin{itemize}
	\item Each LO is encoded as a single gene
	\item The encoding is based on the unique identifier
\end{itemize}
\subsection{The group gene}
\begin{figure}[ht!]
	\centering
	\includegraphics[scale=0.5]{images/group_dimensions.pdf}
	\caption[Group characterization]{Students are characterized in four groups.
		The horizontal axis represents the pre-test score, the vertical axis
		the relevant prior experience.}
	\label{fig:group_dimensions}
\end{figure}
\begin{itemize}
	\item Each student is characterized by two dimensions
		\begin{itemize}
			\item score on the pre-test for this particular knowledge component
			\item relevant experiences
		\end{itemize}
	\item Other things that could have been included are learning
		style, gender and age. As is also common in most user models in
		curriculum sequencing applications. However, in this thesis it
		was decided not to include those. Partly because of the
		difficult way some of the indicators can be retrieved or
		checked automatically, as is for example difficult for learning
		style. And partly because it would enlarge the state space even
		more.
	\item The two dimensions are discretized into two values, high and low.
	\item This results in four categories:
		\begin{description}
			\item[A] This person has a lot of relevant experience but did not
				score well on this particular knowledge component in the
				pre-test.
			\item[B] This person has a lot of relevant experience and also
				scored high on the pre-test for this knowledge component.
			\item[C] This person has neither much relevant experience nor
				a high score on the pre-test for this knowledge component.
			\item[D] This person does not have a lot of relevant experience,
				but did score high on the pre-test on this knowledge component.
		\end{description}
	\item The four categories are also depicted in figure~\ref{fig:group_dimensions}.
	\item These categories are encoded in a gene. This gene functions as a
		match maker. A chromosome containing a particular group gene can only
		be presented to a student that falls in that category.
\end{itemize}
\subsection{The sequence chromosome}
\begin{figure}[h!]
	\centering
	\includegraphics[scale=0.6]{images/concept_hierarchy2.pdf}
	\caption[Components of the chromosome]{Components of the chromosome}
	\label{fig:chromosome_components}
\end{figure}
\begin{itemize}
	\item Refer to figure~\ref{fig:chromosome_components}.
	\item The chromosome contains exactly one group gene
	\item The chromosome contains one or more resource genes.
	\item There should be a bias towards smaller chromosomes.
\end{itemize}
\section{Fitness function}
\begin{itemize}
	\item The literature contains many different implementations of fitness
		functions for the curriculum sequencing problem. 
	\item Some solutions include a term that expresses how well a particuluar
		solution fits the pre-determined prerequisite structure of the learning
		objects. Others include a term that expresses how smooth the transitions
		are between learning objects in difficulty.
	\item A big part of the prerequisite structure is already captured by the
		given main curriculum in this thesis that provides the order of
		knowledge components. The curriculum sequencing step takes place
		in this thesis on a much smaller level where the different learning
		objects all try to convey the same content, but do so in different
		ways. In a way this means that this thesis implements a particular type
		of curriculum sequencing that focuses on the presentation. Some objects
		might contain an example or the formal description. Some might present
		the knowledge in text, others are more visual. Yet it is not expected
		to need only one learning object per student, if you would consider
		the learning object term referring to the smallest unit of explanation.
		Different students require a different mix of presentations at perhaps
		even different orders. Furthermore, learning objects are often not
		a perfect fit. They might explain too much or too little about some
		context. On top of that, it is not that well indexed in terms of the
		exact type of presentation that they have. Thus, what we want is a
		sequence of imperfect learning objects that together maximize the
		educational performance. We do not know what the order should be, given
		that the order is a matter of pedagogy and not knowledge engineering.
		And even if we were able to fully specify the right pedagogical order
		for each type of student perfectly. We would still not have the
		required information about these learning objects, or the information
		might be wrong. In conclusion, we are learning a sequence of black
		boxes of which we only know that they attempt to teach a particular
		knowledge component, or so they say.
	\item The only way we can measure the value of a particular sequence for a
		group of students, and thereby assess its fitness, is to look at the
		gain in knowledge as observed by the post-test.
	\item More precisely the fitness function is the normalized learning gain
		between the pre-test and post-test for a given knowledge component.
\end{itemize}
\section{Population}
\section{Parent selection}
\section{Variation operators}
\subsection{Combination}
\subsection{Mutation}
\section{Survivor selection}
\section{Initialization}
\subsection{Bootstrapping}
\section{Termination condition}
\section{Genetic Algorithm}
