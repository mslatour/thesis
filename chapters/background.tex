\section{Open Educational Resources}
% paraphrase:
%A frequently used definition of  OER is the one provided by one of the key funders of OER initiatives around the world, the William and Flora Hewlett Foundation (Atkins, Brown, and Hammond 2007, 4):
\begin{quote}
``OER are teaching, learning, and research resources that reside in the public domain or have
been released under an intellectual property license that permits their free
use or re-purposing by others. Open educational resources include full courses,
course materials, modules, textbooks, streaming videos, tests, software, and
any other tools, materials, or techniques used to support access to
knowledge.''
\end{quote}
\begin{itemize}
	\item What are open educational resources?
	\item What makes OER different from other sources of educational material
	\item What are the challenges?
\end{itemize}
% there is not that much known about every OER. It costs authors often too much effort, or are otherwise not able, to supply sufficient meta records

\subsection{Lifecycle}
\label{sec:background_oer_lifecycle}
\citet{Collis2004}
enumerates six stages in the lifecycle of a reusable learning object. 

First, a learning object is obtained or created. Second, the learning object is labelled
with metadata information. Third, the learning object is offered to be
selectable for potential use.  Fourth, the learning object is selected
to be used in an educational context. Fifth, the learning object is used either in a
self-contained manner or in combination with other learning objects in an
educational context. Sixth, after of during the learning object is used a
decision is made whether or not to retain this learning object.
%3. Learning objects can be collected in learning
%object repositories, such as ARIADNE \citep{Vidal2004}, in which teachers are
%able to search through the collection for appropriate material. 
%4. , for example by a course developer,
%6. This decision is for example influenced by ratings provided by course developers.
Figure~\ref{fig:oer_lifecycle} depicts the cycle that these stages form.

\begin{figure}[h!]
	\centering
	\includegraphics[width=0.8\linewidth]{images/oer_lifecycle.pdf}
	\caption[Lifecycle of reusable learning objects]{Lifecycle of reusable learning objects, as enumerated by \citep{Collis2004}}
	\label{fig:oer_lifecycle}
\end{figure}
.
%TODO: References!
\subsubsection{Meta data}


\section{Curriculum Sequencing}
\subsection{Reinforcement Learning Approach}
\subsection{Genetic Algorithms Approach}
\subsection{Knowledge Engineering Approach}

\section{Normalized learning gain}
\begin{table}
	\centering
	\begin{tabular}{lll}\hline
		\textbf{Pre-test score} & \textbf{Post-test score} & \textbf{NLG} \\\hline
		0		& 0			& 0 \\
		0		& $^1/_3$	& $^1/_3$\\
		0		& $^2/_3$	& $^2/_3$\\
		0		& 1			& 1\\
		$^1/_3$	& 0			& $-^1/_2$\\
		$^1/_3$	& $^1/_3$	& 0\\
		$^1/_3$	& $^2/_3$	& $^1/_2$\\
		$^1/_3$	& 1			& 1\\
		$^2/_3$	& 0			& $-2$\\
		$^2/_3$	& $^1/_3$	& $-1$\\
		$^2/_3$	& $^2/_3$	& 0\\
		$^2/_3$	& 1			& 1\\
	\end{tabular}
	\caption{Different NLG values}
	\label{tab:nlg_values}
\end{table}


%%%%%%%%%%%%%%%%%%%%%%%%%
% Chapter: Related Work %
%%%%%%%%%%%%%%%%%%%%%%%%%
%\section{Recommender Systems}
%\section{Intelligent Tutoring Systems}
%\section{Adaptive Hypermedia Systems}
%\section{Genetic algorithms}
%There are many different implementations of fitness functions for the
%curriculum sequencing problem~\citep{AlMuhaideb2011}.
%Several systems include a term that expresses how
%well a particular solution fits the pre-determined prerequisite structure of
%the learning objects~\citep{Seki2005, Chen2009, Samia2007}.
%
%The approach presented here takes a different
%approach. A big part of the prerequisite structure is already captured by the
%given main curriculum in this thesis, and as such already provides the order of
%knowledge components. In this thesis, the curriculum sequencing step takes place
%on the lower level of presentation sequencing, where the different learning
%objects all try to convey the same content but do so in different
%ways. Some objects might contain an example or the formal description. Some
%might present the knowledge in text, others are more visual. Different students
%require a different mix of presentations and potentially at different
%orders\footnote{citation needed}.
%
%Other work in this field includes a term that expresses how smooth the
%transitions are between learning objects in difficulty~\citep{Hovakimyan2004,
%Seki2005, Chen2008, Huang2007} or how well their difficulty matches with the
%compentency level of the student\citep{Seki2005, Chen2008, Chen2009, Samia2007, Huang2007}.


