\section{Rules of Nim}
\begin{framed}
\begin{enumerate}
	\item Can you take objects from more than one stack?
		\begin{itemize}
			\item I have no idea
			\item No, you can only take one object at a time
			\item No, you can only take objects from a single stack
			\item Yes, you are allowed to do that
			\item Yes, but only if there are not enough objects on one stack
		\end{itemize}
	\item Bob and John are playing the normal version of a Nim game. John takes
		the last object on table. Who won?
		\begin{itemize}
			\item I have no idea.
			\item Bob won
			\item John won
			\item Nobody won yet
			\item That depens on whether it was John's second turn
		\end{itemize}
	\item How many objects are you allowed to take away from a stack?
		\begin{itemize}
			\item I have no idea.
			\item Exactly one object.
			\item You have to take all the objects of that stack that you chose.
			\item You have to take at least one object.
		\end{itemize}
\end{enumerate}
\end{framed}
\section{Intuition}
\begin{framed}
\begin{enumerate}
	\item On the table are three stacks. The first stack is empty, the second
		stack has two objects and the third stack has one object. What is the
		best move to make?
		\begin{itemize}
			\item I have no idea.
			\item Take one object from the second stack
			\item Take two objects from the second stack
			\item Take one object from the third stack
		\end{itemize}
	\item On the table are three stacks. The first stack has two objects. The
		second stack has two objects. The third stack has one object. What is
		the best move to make?
		\begin{itemize}
			\item I have no idea.
			\item Take one object from the first stack
			\item Take two objects from the first stack
			\item Take one object from the second stack
			\item Take two objects from the second stack
			\item Take one object from the third stack
		\end{itemize}
	\item On the table are three stacks. All stacks have two objects. What is
		the best move to make?
		\begin{itemize}
			\item I have no idea.
			\item Take one object from any single stack
			\item Take two objects from any single stack
		\end{itemize}
\end{enumerate}
\end{framed}
\section{Binary numbers}
\begin{framed}
\begin{enumerate}
	\item What is the binary representation of the decimal number 10?
		\begin{itemize}
			\item I have no idea.
			\item 0010
			\item 1000
			\item 1010
			\item 1111111111
			\item 0000000010
		\end{itemize}
	\item What is the decimal representation of the binary number 1000?
		\begin{itemize}
			\item I have no idea.
			\item 1
			\item 4
			\item 8
			\item 1000
		\end{itemize}
	\item Which number is bigger, the binary number 1001 or the decimal number
		1001?
		\begin{itemize}
			\item I have no idea.
			\item The binary number is bigger.
			\item The decimal number is bigger.
			\item They are equal.
		\end{itemize}
\end{enumerate}
\end{framed}
\section{Nim-Sum}
\begin{framed}
\begin{enumerate}
	\item On the table are three stacks. The first stack has 10 objects. The
		second stack has 8 objects. The third stack has 6 objects. From which
		stack do you take objects when making an optimal move?
		\begin{itemize}
			\item I have no idea.
			\item The first stack.
			\item The second stack.
			\item The third stack.
			\item Any stack will do.
		\end{itemize}
	\item On the table are three stacks. The first stack has 10 objects. The
		second stack has 8 objects. The third stack has 6 objects. How many
		objects will you take from a stack when making an optimal move?
		\begin{itemize}
			\item I have no idea.
			\item Two objects.
			\item Four objects.
			\item Six objects.
			\item Eight objects.
		\end{itemize}
	\item On the table are five stacks. The first stack has 9 objects, the
		second 8, the third 7, the fourth 6 and the fift stack has 5 objects.
		Which move will you make?
		\begin{itemize}
			\item I have no idea.
			\item Nine objects from the first stack.
			\item Eight objects from the second stack.
			\item Seven objects from the third stack.
			\item Six objects from the fourth stack.
			\item Five objects from the fift stack.
		\end{itemize}
\end{enumerate}
\end{framed}
