%%%%%%%%%%%%%%%%%%%%%%%%%%%
% Chapter: Problem Domain %
%%%%%%%%%%%%%%%%%%%%%%%%%%%
\section{Educational context}
\label{sec:task_context}
% TODO: Lesson or knowledge component?
We consider OER sequencing within a lesson of an online course. The course
focuses on a particular topic and is split up into various lessons that cover a
concept or knowledge area that is required for understanding the
course topic. The lesson is taught by an automatic tutor through a sequence of
educational material that is presented to an individual student. Several
educational resources are available for this lesson, from which the tutor needs
to make a selection. The result of this selection is an ordered sequence of
educational resources that are consecutively presented. At the beginning and at
the end of each lesson there is an assessment, refered to as the pre-test and
post-test respectively. These assessments measure the relevant competence level
befor and after the presented sequence. The situation is depicted in
Figure~\ref{fig:course_lesson_diagram}. The task central to this thesis is to
find the optimal sequence of resources.\\
\begin{figure}[ht]
	\centering
	\includegraphics[width=0.8\linewidth]{images/course_lesson_diagram.pdf}
	\caption[Educational context of the task]{The sequencing task takes place
	within a lesson of a course.}
	\label{fig:course_lesson_diagram}
\end{figure}\\
\noindent
There is no pre-defined length of the sequences for each lesson. Only one
educational resource could be enough to explain it. The lesson might also
require multiple resources. However, the lengths are restricted to a
pre-defined lower and upper bounds. Additionally, to simplify the problem
slightly more, the sequence must not contain an educational resource multiple
times. This is partly to reduce the number of possibilities and partly because
the time span between the duplicates would be rather short. In this short time
span, it is unlikely that repetition would be useful.
\subsection{Quality of OER}
The quality of a particular sequence of educational material is determined by
the impact it has on learning. The impact is defined as the measured normalized
learning gain between the pre-test and the post-test, as given by
Equation~\eqref{eq:nlg}. A sequence is optimal when the measured normalized
learning gain is maximized. The search task is thus straightforward. Namely
finding the sequence with maximum impact out of all possible sequences.
Equation~\eqref{eq:optimality} describes this mathematically, where $R$
denotes the set of all educational resources available for that lesson, $C^1$
and $C^2$ denote the normalized pre-test and post-test scores respectively, $a$
and $b$ denote the minimum and maximum length of a sequence respectively and
$S$ denotes the set of all possible sequences for that lesson.
\begin{equation}
	\label{eq:optimality}
	\argmax_{s \in S} \left[ nlg(C^1, C^2_s) \right],
	\quad S = \bigcup_{k = a}^{b} R^k
\end{equation}
\subsection{Solution space}
The number of sequences of educational resources that need to be considered can
increase quickly. A sequence of three slots can be instantiated from a
collection of five resources in 60 ($5*4*3$) different ways. When the
collection is twice as large, in total 720 sequences would be possible. The
number of possible sequence instantiations is given by the \emph{k-permutation
of n elements}, where \emph{k} is the number of slots to instantiate and
\emph{n} the number of elements to draw from.\\\\
\noindent
Apart from the number of different instantiations, sequences have a variable
length.  The total number of possible sequences is thus the sum of all possible
instantiations of sequences of all possible lengths. This results in
Equation~\ref{eq:search_space_size}, where $|R|$ denotes the number of
resources for the lesson, $a$ and $b$ denote the minimum and maximum number of
resources in the sequence respectively and $|S|$ denotes the number of possible
sequences. The equation is essentially the formula for the number of
\emph{k-permutations of n} given by $\tfrac{n!}{(n-k)!}$ summed for all
possible values of \emph{k}.
Table~\ref{tab:search_space_sizes} shows the outcome of this 
equation for various values of $|R|$, $a$ and $b$.
\begin{equation}
	\label{eq:search_space_size}
	|S| = \sum_{k=a}^{b} \frac{|R|!}{(|R|-k)!}, \quad a \le b \le |R|
\end{equation}
\begin{table}[ht]
	\begin{minipage}{0.24\linewidth}
	\centering
	\begin{tabular}{llll}\hline
		\textbf{$|R|$} & \textbf{$a$} & \textbf{$b$} & \textbf{$|S|$} \\\hline
		3 & 1 & 3 & 15 \\
		4 & 1 & 3 & 40 \\
		5 & 1 & 3 & 85 \\
	\end{tabular}
	\end{minipage}
	\begin{minipage}{0.24\linewidth}
	\centering
	\begin{tabular}{llll}\hline
		\textbf{$|R|$} & \textbf{$a$} & \textbf{$b$} & \textbf{$|S|$} \\\hline
		5 & 1 & 4 & 205 \\
		5 & 1 & 5 & 325 \\
		5 & 2 & 5 & 320 \\
	\end{tabular}
	\end{minipage}
	\begin{minipage}{0.24\linewidth}
	\centering
	\begin{tabular}{llll}\hline
		\textbf{$|R|$} & \textbf{$a$} & \textbf{$b$} & \textbf{$|S|$} \\\hline
		10 & 1 & 3 & 820 \\
		10 & 1 & 4 & 5860 \\
		10 & 2 & 4 & 5850 \\
	\end{tabular}
	\end{minipage}
	\begin{minipage}{0.26\linewidth}
	\centering
	\begin{tabular}{llll}\hline
		\textbf{$|R|$} & \textbf{$a$} & \textbf{$b$} & \textbf{$|S|$} \\\hline
		10 & 3 & 5 & 36000 \\
		10 & 1 & 10 & 9.864.100 \\
		10 & 5 & 10 & 9.858.240 \\
	\end{tabular}
	\end{minipage}
	\caption[Solution space size]{Solution space sizes calculated for a few parameter values using
		Equation~\eqref{eq:search_space_size}. Parameters $|R|$, $a$ and $b$ represent
	the number of resources, the minimum length of a sequence and the maximum
length of a sequence respectively. $|S|$ represents the number of possible
sequences, given the parameters.}
	\label{tab:search_space_sizes}
\end{table}
\subsection{Student Groups}
\label{sec:task_student_groups}
Within the scope of this thesis, the sequences are not optimized for every
individual student. Instead, each student is assigned to a student group, for
which the sequence of educational material is optimized. Students are assigned
to the student group based on their normalized pre-test score. The score is
descretized in a low and high value.  The optimal sequence of education
material must be found for each student group. In order words, students
who know little about the topic might receive different material than students
who know a lot. This results in a new situation, which is depicted by
Figure~\ref{fig:student_group_sequencing}.\\
\begin{figure}[ht]
	\centering
	\includegraphics[width=0.7\linewidth]{images/student_group_sequencing.pdf}
	\caption[Educational context of the task with student groups]{The
	sequencing task is performed separately for each student group.}
	\label{fig:student_group_sequencing}
\end{figure}\\
\noindent
There are several other features on which students could have been
divided even further into more specific groups. Learning style, gender and age
are not uncommon feature candidates. These are not used in this thesis for
mainly two reasons. First, determining the values for these characteristics can
be difficult and unreliable in a web context. Second, this would result in more
groups. Each additional group requires new students in order to find the
optimal sequence in that group. Acquiring many students is not possible within
this thesis.\\\\
\noindent
The coarse division of students in groups could cause a large diversity of
students within each group. That means that the evaluation of a sequence by a
student is not necessarily representative for the average evaluation. On top of
that, the assessment of impact will be noisy. For example because students
might be distracted during one of the assessments. That means that sequences
must be evaluated multiple times in order to acquire a better estimate.\\\\
\noindent
Although the optimal sequence is determined per student group, they are not
entirely independent. This becomes apparent through an analogy. A teacher that
teaches different cohorts will try to optimize his teaching for each
separate group. If however the teacher would observe that a particular approach
works really well for one group, the teacher might try it out on other groups
as well. The underlying assumption being that there are some universal do's and
don'ts when explaining a certain topic. Thus, students in different student
groups are likely to have differences as well as similarities.

\section{Multiple objectives of OER sequencing}
Evaluating the optimality of a sequence comes with a cost.
Each evaluation requires a new student and there are only limited
students available. Therefore the number of students required to find the
optimal sequence need to be minized. Furthermore, presenting a
less-than-optimal sequence to a student could limit the amount of learning.
That is something one wants to avoid in an educational setting.\\\\
\noindent
The ``damage'' done to the student's learning can be expressed in
\emph{regret}. The definition of regret is the difference in the reward
received between performing the optimal action and the current action. In other
words how much reward is missed by not choosing the optimal action. In the
context of this thesis, regret is the learning gain that is missed by not
presenting the optimal sequence.\\\\
\noindent
However, we do not know beforehand what the optimal sequence is. The process of
searching for the optimal sequence will result in evaluations of
less-than-optimal sequences. Recall that these evaluations involve actual
students. Thus, the regret built up during this search process, known as the
\emph{online regret}, needs to be minimized. Furthermore recall that
observations of the sequence's optimality are noisy. Multiple evaluations are
needed for accurate estimates. This results in two objectives, namely
better estimates and minimizing online regret.\\\\
\noindent
This situation is familiar to many fields and is known as the exploration vs.
exploitation trade- off \citep{Holland1992}. An often used example is the
n-armed bandit problem \citep{Sutton1998} where a casino offers n different
slot machines (a.k.a. one-armed bandits) to play. A player would want to
optimize the total amount of money earned and is therefore looking for the slot
machine that has the highest pay-off. It is tempting to stay with a slot
machine that gives the highest return you have seen so far (exploitation), but
it is important to also try out other slot machines to see if they perform even
better (exploration). This can be applied to OER sequencing by aligning the
bandits with sequences of learning objects and the pay-off with learning gain.
%This thesis focusses on the utilization of Open Educational Resources (OER) in the
%lessons, which means the optimization procedure must be able to adapt to new
%resources being added while running. For the scope of this thesis, the event of
%an OER dissapearing will not be taken into account.
