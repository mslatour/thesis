%%%%%%%%%%%%%%%%%%%%%%%%%%%
% Chapter: Problem Domain %
%%%%%%%%%%%%%%%%%%%%%%%%%%%
\label{ch_problem_domain}
% TODO: Completely restructure, give curriculum sequencing a much greater part
\begin{itemize}
	\item Introduce curriculum sequencing
		\begin{itemize}
			\item generate personalized learning path through content (cite
				brusilovsky)
			\item old feature, recent necessity due to hyperspace. OER
				especially!
		\end{itemize}
	\item Typically fitting content to students, due to differences in student
		ability or learning style. OER adds a challenge of also dealing with
		bad content.
	\item Furthermore several examples of curriculum sequencing in the
		literature that apply evolutionary computing focus on creating a
		content sequencing with a smooth built-up in dificulty. However, in OER
		this difficulty is not necessarily known. And if it is, there is no
		guarantee that it will be correct.
\end{itemize}
\begin{figure}[ht!]
	\centering
	\includegraphics[scale=0.8]{images/concept_hierarchy.pdf}
	\caption[Concept hierarchy]{The hierarchy of concepts in this educational
	task}
	\label{fig:concept_hierarchy}
\end{figure}
\begin{itemize}
	\item refer to the different concepts that are used throughout the thesis,
		as depicted in \ref{fig:concept_hierarchy}
\end{itemize}
\section{Curriculum}
\subsection{Knowledge Components}
\section{Open Educational Resources}
\begin{itemize}
	\item What are open educational resources?
	\item What makes OER different from other sources of educational material
	\item What are the challenges?
\end{itemize}
\subsection{Meta data}
\section{Pre-test / post-test} % domain-agnositic? what about prior math experience
\section{Search space}
\begin{itemize}
	\item let R be number of resources for a topic
	\item let A be the minimum number of resources in a sequence
	\item let B be the maximum number of resources in a sequence
	\item point to Equation~\eqref{eq:search_space_size}
	\item point to Table~\ref{tab:search_space_sizes} for some example values
\end{itemize}
\begin{equation}
	\label{eq:search_space_size}
	|S| = \sum_{i=0}^{B-A} \frac{R!}{(R-B+i)!}, \quad A \le B \le R
\end{equation}
\begin{table}
	\begin{minipage}{0.24\linewidth}
	\centering
	\begin{tabular}{l|l|l||l}
		\textbf{R} & \textbf{A} & \textbf{B} & \textbf{$|S|$} \\\hline\hline
		3 & 1 & 3 & 15 \\\hline
		4 & 1 & 3 & 40 \\\hline
		5 & 1 & 3 & 85 \\\hline
	\end{tabular}
	\end{minipage}
	\begin{minipage}{0.24\linewidth}
	\centering
	\begin{tabular}{l|l|l||l}
		\textbf{R} & \textbf{A} & \textbf{B} & \textbf{$|S|$} \\\hline\hline
		5 & 1 & 4 & 205 \\\hline
		5 & 1 & 5 & 325 \\\hline
		5 & 2 & 5 & 320 \\\hline
	\end{tabular}
	\end{minipage}
	\begin{minipage}{0.24\linewidth}
	\centering
	\begin{tabular}{l|l|l||l}
		\textbf{R} & \textbf{A} & \textbf{B} & \textbf{$|S|$} \\\hline\hline
		10 & 1 & 3 & 820 \\\hline
		10 & 1 & 4 & 5860 \\\hline
		10 & 2 & 4 & 5850 \\\hline
	\end{tabular}
	\end{minipage}
	\begin{minipage}{0.26\linewidth}
	\centering
	\begin{tabular}{l|l|l||l}
		\textbf{R} & \textbf{A} & \textbf{B} & \textbf{$|S|$} \\\hline\hline
		10 & 3 & 5 & 36000 \\\hline
		10 & 5 & 10 & 9.858.240 \\\hline
		10 & 1 & 10 & 9.864.100 \\\hline
	\end{tabular}
	\end{minipage}
	\caption[Search space size]{Search space sizes calculated for a few parameter values using
		equation~\eqref{eq:search_space_size}. Parameters R, A and B represent
	the number of resources, the minimum length of a sequence and the maximum
length of a sequence respectively.}
	\label{tab:search_space_sizes}
\end{table}

