%%%%%%%%%%%%%%%%%%%%%%%%%%%
% Chapter: Problem Domain %
%%%%%%%%%%%%%%%%%%%%%%%%%%%
\label{ch_problem_domain}
% TODO: Lesson or knowledge component?
The work presented in this thesis aims to contribute to solving a specific
task, which is part of the broader category of curriculum sequencing tasks.
The task is situated within a lesson of an online course. The course
focuses on a particular topic and is split up into various lessons that should
cover a concept or knowledge area that is required for understanding the
course topic. The lesson is taught by an automatic tutor through a set of
educational material that is presented to an individual student. Several
educational resources are available for this lesson, from which the tutor can
make a selection. The result of this selection is an ordered sequence of
educational resources that are consecutively presented. The task central
to this thesis is to find the optimal sequence of resource`s.

The optimality for these sequences is defined to be evicence-based. At the
beginning of the lesson there is a pre-test, where the student's prior knowledge
regarding the subject of the lesson is assessed. At the end of the lesson,
after the sequence of content has been presented, a post-test will take place.
In the post-test the student's relevant knowledge level will be assessed again.
The sequence is said to be optimal when the measured normalized learning gain
is maximized. Sequences can be of varying length, only restricted to some pre-set
lower and upper bounds. Additionally, to simplify the problem slightly more,
the sequence must not contain an educational resource multiple times. This is
partly to reduce the number of possibilities and partly because the time span
between the duplicates would be rather short, making it unlikely to be useful.

\begin{framed}\noindent
\textbf{{\large Definition:}} The task at hand. \vspace{0.5\onelineskip} \hrule
\vspace{\baselineskip}\noindent
Find the sequence of educational resources within the context of a lesson that
maximizes the normalized learning gain for a student, measured as a
result of presenting the sequence to that student.
Equation~\eqref{eq:task} describes this mathematically, where $R$ denotes the
set of all educational resources available for that lesson, $C^1$ and $C^2$
denote the normalized pre-test and post-test scores respectively, $a$ and $b$
denote the minimum and maximum length of a sequence respectively and $S$
denotes the set of all possible sequences for that lesson.\noindent
\begin{equation}
	\label{eq:task}
	\argmax_{s \in S} \left[ \frac{C^2_s-C^1}{1-C^1} \right],
	\quad S = \bigcup_{k = a}^{b} R^k
\end{equation}
\end{framed}

This thesis focusses on the utilization of Open Educational Resources (OER) in the
lessons, which means the optimization procedure must be able to adapt to new
resources being added while running. For the scope of this thesis, the event of
an OER dissapearing will not be taken into account.

The rest of this chapter is focused as follows.
Section~\ref{sec:task_student_groups} describes the division of students into
student groups, but first the search space is described in
Section~\ref{sec:task_search_space}.

\section{Search space}
\label{sec:task_search_space}
The curriculum sequencing task is a search problem, with the aim of searching the
space of all valid sequences to find the optimal one. A valid sequence is a
sequence without duplicates with a number of elements that falls within the
boundaries. These two aspects determine the number of possible sequences. 
The fact that a sequence doesn't have duplicates means that a
sequence of three slots can be instantiated from a collection of five resources in
60 ($5*4*3$) different ways, since the first, second and third slot can be
instantiated in five, four and three ways respectively. This is formally a
\emph{k-permutation of n elements}, where \emph{k} is the number of slots to
instantiate and \emph{n} the number of elements to draw from.
If the sequence can have multiple lengths, then the number of
possibilities are found by adding the
number of possibilities for each length together. This results in
Equation~\ref{eq:search_space_size}, where $|R|$ denotes the number of
resources for the lesson, $a$ and $b$ denote the minimum and maximum number of
resources in the sequence respectively and $|S|$ denotes the number of possible
sequences. The equation is essentially the formula for the number of
\emph{k-permutations of n} given by $\tfrac{n!}{(n-k)!}$ summed for all
possible values of \emph{k}.
Table~\ref{tab:search_space_sizes} shows the outcome of this 
equation for various values of $|R|$, $a$ and $b$.
\begin{equation}
	\label{eq:search_space_size}
	|S| = \sum_{k=a}^{b} \frac{|R|!}{(|R|-k)!}, \quad a \le b \le |R|
\end{equation}
\begin{table}[ht]
	\begin{minipage}{0.24\linewidth}
	\centering
	\begin{tabular}{llll}\hline
		\textbf{$|R|$} & \textbf{$a$} & \textbf{$b$} & \textbf{$|S|$} \\\hline
		3 & 1 & 3 & 15 \\
		4 & 1 & 3 & 40 \\
		5 & 1 & 3 & 85 \\
	\end{tabular}
	\end{minipage}
	\begin{minipage}{0.24\linewidth}
	\centering
	\begin{tabular}{llll}\hline
		\textbf{$|R|$} & \textbf{$a$} & \textbf{$b$} & \textbf{$|S|$} \\\hline
		5 & 1 & 4 & 205 \\
		5 & 1 & 5 & 325 \\
		5 & 2 & 5 & 320 \\
	\end{tabular}
	\end{minipage}
	\begin{minipage}{0.24\linewidth}
	\centering
	\begin{tabular}{llll}\hline
		\textbf{$|R|$} & \textbf{$a$} & \textbf{$b$} & \textbf{$|S|$} \\\hline
		10 & 1 & 3 & 820 \\
		10 & 1 & 4 & 5860 \\
		10 & 2 & 4 & 5850 \\
	\end{tabular}
	\end{minipage}
	\begin{minipage}{0.26\linewidth}
	\centering
	\begin{tabular}{llll}\hline
		\textbf{$|R|$} & \textbf{$a$} & \textbf{$b$} & \textbf{$|S|$} \\\hline
		10 & 3 & 5 & 36000 \\
		10 & 1 & 10 & 9.864.100 \\
		10 & 5 & 10 & 9.858.240 \\
	\end{tabular}
	\end{minipage}
	\caption[Search space size]{Search space sizes calculated for a few parameter values using
		Equation~\eqref{eq:search_space_size}. Parameters $|R|$, $a$ and $b$ represent
	the number of resources, the minimum length of a sequence and the maximum
length of a sequence respectively. $|S|$ represents the number of possible
sequences, given the parameters.}
	\label{tab:search_space_sizes}
\end{table}


\section{Student Groups}
\label{sec:task_student_groups}
Within the scope of this thesis, the sequences are not optimized for every
individual student. Instead, each student is assigned to a student group, for
which the sequence of educational material is optimized. There are several
characteristics on which students can be divided into groups. Learning style,
gender or age of students are commonly used in curriculum sequencing
applications\footnote{Citation needed}. Another suggestion could be relevant
experience, for example having played a lot of other mathematical games is
relevant to the experiment conducted in this thesis. However, these are not
used in this thesis. Partly because determining the values for these
characteristics can be difficult and unreliable in a web context, but mostly
because it would require much more students in the experiment in order to train
all these student groups. In this thesis only one dimension is used to divide
students in two groups, which is the prior knowledge a student has about a
particular knowledge component as measured by the pre-test. The pre-test score
is discretized in two values, high and low.

The curriculum sequencing problem is split up per student group,
while maintaining the ability for good solutions to be copied.
Analogous to a teachter that teaches in different grades, the algorithm
tries to optimize the learning material for each group separately. But
in that same analogy, if the teacher would see that a particular
approach would work really well for one group he might be inclined to
try it out on the other groups as well. The underlying assumption is
that there are some universal do's and don'ts for the explanation of
the same topics, some examples work better than others etc. The student
groups, although different on a few selected dimensions, are likely to
be similar on many other dimensions.


