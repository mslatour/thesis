%%%%%%%%%%%%%%%%%%%%%%%%%
% Chapter: Introduction %
%%%%%%%%%%%%%%%%%%%%%%%%%
% 1. Open Education is a big trend,
%    which also causes an increasing interest in Open Educational Resources
Open Education is a big theme in Higher Education, both in scientific
\citep{Wiley2014, Duval2010} and practice oriented communities
\citep{Bussemaker2014, Trendsurfoe2014, Yuan2013, Johnson2013, OECD2007}.
The Open Educational Resources (OERs) movement, which is a significant
part of this theme, aims to ``provide open access to high quality digital
educational materials'' \citep{Caswell2008}. Starting with MIT's
OpenCourseWare initiative \citep{Abelson2008} in 2001, the movement has grown
into a global topic of conversation \citep{Hylen2012}.
Open Educational Resources are a type of learning objects, which are defined as
``any reusable digital resource that is encapsulated in a lesson or assemblage
of lessons grouped in units, modules, courses, and even programmes''
\citep{McGreal2004}. The two terms are in the context of this thesis mostly
interchangeable. Both address resources that are available to be reused in a
learning context. An important difference however, is that an Open Educational
Resource is also free\footnote{Free as in free speech, not necessarily as in
free beer} to be revised (i.e. altered), remixed (i.e. combined with others)
and afterwards redistributed \citep{Hilton2010}. This effectively lowers the
threshold for innovation with educational material, which facilitates at least two
things. One, the creation and distribution of more
(derivatives of) educational content. Two, a larger diversity in authors, which
results in more a diverse quality of the shared learning objects
\citep{Weller2010}.\\\\
\noindent
% 2. OER assigns vital role to teachers
Section~\ref{sec:background_oer_lifecycle} describes six stages in the
lifecycle of a reusable learning object. All of these stages involved some form
of human activity, either by content developers or instructors\footnote{Obviously not
disjoint groups}. An obvious example of that would be the actual creation and
labelling of content. However, more interestingly, the selection of learning
objects created by others and the subsequent determination of their quality are
also particularly dependent on actions of instructors.
\noindent
% 3. The quantity is too big to continue this dependency on teachers
One could question how sustainable this is.
\citet{Ochoa2009} show that the size of learning object collections
vary from hundreds to millions of objects depending on the type of
repository. \citeauthor{Ochoa2009} further state that almost all examined
repositories are growing linearly at the moment. However,
\citeauthor{Ochoa2009} expect that exponential growth
will occur when the repositories are capable of retaining its productive users.
With this increasing number of open educational resources it becomes
impractical to determine the quality of learning objects using human effort
\citep{Cechinel2011}. Ratings were for example used as primary indicator of
quality in the MERLOT repository, but only a small proportion was actually
rated by peers \citep{Ochoa2006, Zemsky2004}.\\\\
\noindent
% 4. Therefore, there is a quest towards automatic means of quality assessment.
%    Several approaches are on the way, but all leave incorporating actual
%    learning gain to teachers
As a result, the interest in automatic assessments of learning object quality
increased. One approach focuses on the quality of the metadata of the learning
object \citep{Ochoa2009, Tani2013}, as a proxy for the quality of the object
itself.
% TODO: does this even make sense in the story?
A problem with this approach is that metadata can be inaccurate
\citep{Cechinel2009} and incomplete \citep{Sicilia2005}.
% and it says nothing about the actual quality of the learning object it self.
A different approach
is to automatically assess the quality of learning objects based on intrinsic
metrics, such as the number of links in a learning object.
\citeauthor{Cechinel2011} managed to find statistical profiles of different
quality labels by comparing intrinsic metrics of good and poor learning objects
in the MERLOT repository. \citet{Duval2006} however states that the quality of
a learning object is context-dependent and intrinsically subjective. As
examples of context \citeauthor{Duval2006} mentions a.o. learning goal,
available time and educational level. \citeauthor{Duval2006} proposes the
context-dependent ranking algorithm  LearnRank, in which learning objects that
are used in many contexts receive a higher rank. In \citep{Ochoa2006} the ``use
of contextualized attention metadata for ranking and recommending learning
objects'' is proposed, where the attention given to a learning object by a
student or a teacher within a certain context is considered to be a proxy for
the usefulness of the object in that context. \citeauthor{Duval2006} however
also mentions in \citep{Duval2006} that
``In an ideal world, we would actually bootstrap and steer this process [of
establishing the LearnRank] through empirical data on the learning effect that
specific objects have actually caused (or helped to realise) in specific
contexts\ldots''. According to \citep{Kay2007} the effect of a learning
object on 'actual learning' is underrepresented in the research on learning
objects. In a recent report from the European Commission, quality of Open
Educational Resources had five aspects: efficacy, impact, availability,
accuracy and excellence \citep{Camilleri2014}. The aspect of impact referred to
the ``extend to which an object or concept proves effective''. However, little is
said in \citet{Camilleri2014} about methods that would measure
this impact. \citeauthor{Camilleri2014} further state as one of the challenges
that educational resources of high quality are fragmented with no particular way of
distinguishing them from the other available learning objects. They recommend
creating ``specialised directories'' where exclusive lists of repositories of
high-quality content are maintained. It is however not stated how these
high-quality repositories could incorporate impact in their quality assessment.\\\\
\noindent
% 6.There is however growing interest in evidence-based decisions
% 7. Learning objects are typically presented as part of a sequence in a LMS
% 8. And in fact the sequence influences, as being part of the context,
%    each individual learning object’s quality
% 9. The assessments in the LMS could provide insight in the learning gain,
%    as a result of the sequence
At the same time, there is a growing interest in the use of
quantitative data for evidence-based teaching decisions \citep{Wayman2005,
Marsh2006, Spillane2012, Clow2013}. The electronic learning environments where the
educational material is presented to learners often also contain
digital assessments of the learning outcomes linked to that educational
material. It is important to note however, that learning objects are
first sequenced with other learning objects before presented to students
\citep{Brusilovsky1992, Quinn2000}. The activities before and after a
learning object also affect its quality \citep{Duval2006}. It therefore makes
sense to determine the quality of a learning object within the sequence it is
part of. The coexistence of such a sequence and digital assessments of learning
allows for the collection of empirical evidence of the sequence quality
\citep{Latour2014}. The situation that would make this possible is shown in
Figure~\ref{fig:assessment_of_sequences} where a sequence of OER is embedded in
between two assessments. The first assessment would provide a benchmark of a
student's knowledge, with which the result of the second assessment could be
compared. The comparison would in turn provide empirical evidence for the
quality of the sequence used.
\begin{figure}[h!]
	\begin{framed}
	\centering
	\includegraphics[width=0.9\linewidth]{images/assessment_of_sequences.pdf}
	\caption[Coexistence of learning objects and assessments]{The coexistence of digitial assessments and sequences of learning
	objects can result in a situation where the sequence is embedded in between
	two assessments.}
	\label{fig:assessment_of_sequences}
	\end{framed}
\end{figure}\\\\
\noindent
This approach is however not trivial. Even when the number of relevant learning
objects is relatively small, there will still be a large number of sequences
possible. Section~\ref{sec:task_search_space} provides a formula to calculate
this number. Each of these sequences will require multiple evaluations due to
the inherent noise in the domain. This noise for example comes from the fact
that the test results of one learner will not be representative of the average
results of an entire group. This is not only due to the fact that learners have
different backgrounds and skills to start with, and as such do not form a
uniform group. Education in general is a complex process, of which it is likely
that only a part takes places within the observable electronic learning
enviroment. This means that a sequence of learning material that appears to
have no significant effect on the learning process of a particular learner must
still be presented to a few more students in order to be more sure about its
quality. However, this experimentation clearly does not come without cost. Each
time a sequence shows to be less effective than a different one, a learner has
received a lower level of education than necessary.\\\\
\noindent
This dilemma is familiar to many fields and is known as the \emph{exploration
vs. exploitation trade-off} \citep{Holland1992}. An often used example is the
\emph{n-armed bandit problem} \citep{Sutton1998} where a casino offers \emph{n}
different slot machines (a.k.a. one-armed bandits) to play. A player would want to
optimize the total amount of money earned and is therefore looking for the slot
machine that has the highest pay-off. It is tempting to stay with a slot
machine that gives the highest return you have seen so far (exploitation), but
it is important to also try out other slot machines to see if they perform even
better (exploration). This can be related to the domain of open education by
aligning the bandits with sequences of learning objects and the pay-off
with learning gain. A teacher would then be required to select from the
metaphorical casino floor the sequences to try in order to optimize the total
learning gain of his or her students. A task that is far from trivial to do.
\\\\
\noindent
An alternative approach would be to automatically construct the sequences of
learning material, a task known as curriculum sequencing \citep{AlMuhaideb2011}.
In particular, due to the ever-growing nature of open educational resources,
the automatic mechanism would have to deal with the open corpus problem known
in the field of adaptive educational hypermedia \citep{Brusilovsky2007}.
Such a mechanism could address the desire for empirical evidence
of quality of learning objects on the one hand, while also addressing the
difficulties of obtaining that evidence from interactions in a real learning
context. The mechanism could be a component in learning object repositories
that suggests learning object sequences to teachers, while collecting data from
the learning environment that the teacher uses. It could also
function as a component in a fully automatic intelligent tutoring system
\citep{Psotka1988}, which replaces the role of the teacher.\\\\\
\noindent
% state that the work presented in this thesis is such a mechanism.
% explain thesis work
The work presented in this thesis is such a mechanism.
This thesis introduces and evaluates a novel approach for
the open corpus problem in intelligent tutoring systems, which is especially
applicable to the usage of Open Educational Resources.
In its core, the developed system applies genetic algorithms to construct
sequences of learning material based on the measured learning gain it caused,
while balancing exploration and exploitation using Upper Confidence Bound (UCB)
selection.\\\\
\noindent
This thesis is structured as follows. Chapter~\ref{ch:background} provides
background information and related work. Chapter~\ref{ch:curriculum_sequencing}
contains a detailed description of the task. The approach is described in
Chapter~\ref{ch:approach}, the resulting software is described in
Chapter~\ref{ch:software}. In Chapter~\ref{ch:simulations} a series of
simulations are discussed. Chapter~\ref{ch:experimental_setup} covers the setup
of the experiment. The results of this experiment are discussed in
Chapter~\ref{ch:results}. Chapter~\ref{ch:discussion_conclusion} concludes the
findings in this thesis and provides general discussion. Finally,
Chapter~\ref{ch:future_work} enumerates some future work.
