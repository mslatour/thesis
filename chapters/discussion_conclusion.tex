%%%%%%%%%%%%%%%%%%%%%%%%%%%%%%%%%%
% Chapter: Discussion/Conclusion %
%%%%%%%%%%%%%%%%%%%%%%%%%%%%%%%%%%

In this thesis a novel approach to automatic assessment of OER quality has been
presented and evaluated. Measurements of the impact of OER on learning as
component of the OER quality was largely missing in the literature. The
\emph{TutOER} system evaluates OER in a sequence by assessing students' competence levels
before and after the presentation of the OER sequence . The confidence of the estimate
had to be balanced with the cost of presenting bad educational content to
students. The curriculum sequencing task from the intelligent tutoring systems
community has been used as framework to do this. The UCB-1 selection mechanism
was applied to determine which OER te evaluate. A genetic algorithm was used
to limit the scope of UCB-1 and to have a more directed and structured
search.\\\\
\noindent
The \emph{TutOER} system was evaluated in two ways. First, a series of
simulations were performed. Each simulation explored a combination of parameter
values for the genetic algorithm. Second, an online experiment was conducted
in which the participants took a course on the mathematical game Nim. The
course consisted of four lessons. In each lesson a sequence of OER was
presented to the participant. The competence level of the student was assessed
at the beginning and end of each lesson. The measure of impact was defined as
the normalized learning gain between the two assessments.\\\\
\noindent
The experiment results show that fitness evaluations can vary a lot for one
sequence. There are three possible causes for that. Firstly, the assessments
presented in the experiment had only three multiple-choice questions. The
results from these assessments are thus more susceptible for noise than a more
extensive assessment. In particular, potential guessing of students has a high
impact on the outcome. Secondly, the division of students in student groups is
rather coarse. It is highly likely that students within one student group form
everything but a homogeneous group. Not only did students who had zero
questions or one question correct in the pre-test end up in the same group.
Students also potentally differed on learning style, age or prior knowledge on
related topics. Not to mention that students who participated presumingly did
so in different environments, levels of concentration, committment and time to
spare. These factors could all have an impact on the measured learning gain,
especially in the more complex lessons at the end. Lastly, the system assumes
that the performance on the current sequence is independent off the sequences
presented in other lessons. This assumption is of course a simplification of
reality, since whether a student understands binary numbers will affect the
chance of learning the nim-sum. Part of the noise in the fitness evaluations
could be explained by the fact that students in the same group had different
learning experiences in previous lessons.\\\\
\noindent
Despite the varying fitness observations, the \emph{TutOER} was capable of
finding the best sequence in at least four populations in the experiment. The
best sequences had on average a better fitness than the other sequences. In some
cases the best sequence appears promising enough to be close to a global
optimum. For example, the sequence found in the \emph{Rules low} population had
a high fitness of 0.7634. Furthermore, 103 out of 178 students got all
post-test question correct after seeing this sequence. Many more students at
least improved their score.\\\\
\noindent
However, in some other cases the best sequence is less convincing. For example,
the best sequence in the \emph{Nim-sum low} population had a low fitness of
0.2557. More than half of the students had an equal or even lower post-test
score compared to their pre-test. In these cases it is not a stretch to imagine
that a better sequence might have existed. Even though on average this sequence
was better than the rest on decisive moments. Due to the chosen parameter
values of the genetic algorithm, decisions are made on relatively few
evaluations. Furthermore, the order in which the students participate is
approximately random. Given that most if not all OER sequences will perform bad
with a few students. It could happen that by chance a sequence would get those
few students at the beginning. A different sequence might get those few
students at the end, and will perform better on average at the beginning in
comparison. Due to the directed search of the genetic algorithm, it might take
some time before a sequence is granted enough opportunities to correct the
quality estimate.\\\\
\noindent
This is of course true for a normal teacher evaluating educational material as
well. In fact, it is true for every one-armed bandid situation where there is a
exploration vs exploitation trade-off in a noisy environment. The solution
appears straightforward, but it comes with a cost. The solution is to evaluate
sequences more often before making a selection. The parameter that influences
that directly is the number of episodes. That parameter was set to a low value
for the experiment, which also becomes clear from the results. Raising it would
however also inevitably raise the regret built-up from sequences that turn out
to be bad. A more indirect parameter is the number of individuals in the
population. A larger population would increase the chance of any individual to
be selected. Although in this thesis, sequences that caused a negative normalized
learning gain had a zero chance of being selected. More diversity in the
population would slow down the convergence of the genetic algorithm. However,
due to UCB-1 selection, the exploration versus exploitation trade-off would be
better managed.\\\\
\noindent
The results do show that it is possible to take into account the impact that
OER has on learning. Even when that impact is unknown at the beginning. In the
future, a feedback loop could be introduced from a learning environment where
the OER is deployed back to learning object repository. Gathering data from
many students will allow for more specific student categories. The exploration
towards new materials could be handled by recommender systems and intelligent
tutors. In a way similar to the work presented in this thesis.
