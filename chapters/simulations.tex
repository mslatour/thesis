%%%%%%%%%%%%%%%%%%%%%%%%
% Chapter: Simulations %
%%%%%%%%%%%%%%%%%%%%%%%%
Apart from a modeling of the domain, genetic algorithms also require several
parameter values to be set. These values can, together with modeling decisions,
have an enormous impact on peformance. It is therefore important to gain more
insight in the properties and effectiveness of the genetic algorithm approach
under various parameter values. This chapter describes and analyses a series of
simulations that explored this. The goal of the simulations is to find good
parameter values for the experiment. Additionally they provide some insight in
the behavior and performance of the \emph{TutOER} system after a larger number
of evaluations that would be possible in an experiment. The simulations are
executed using the same software that is used for the experiment. However, for
the sake of simplicity and speed the actual web interaction was left out of the
simulation. Sequences are therefore evaluated using a handcrafted model instead
of using actual students. Each simulation is repeated ten times to deal with
the random components of the algorithm.\\\\
\noindent
This chapter is organized as follows. The parameters that are examined are
described in Section~\ref{sec:simulations_parameters}.  The simulations that
cover these parameters are enumerated in
Section~\ref{sec:simulations_simulations}. Section~\ref{sec:simulations_setup}
describes the general setup of each simulation. In
Section~\ref{sec:simulations_results} the simulation results are listed and an
analysis presented of the findings.

\section{Parameters}
\label{sec:simulations_parameters}
The approach described in Chapter~\ref{ch:approach} requires several parameters
to be set in advance. These parameters influence the behavior and performance
of the algorithm and are often not independent. This section describes the
effect of the four parameters that need to be set: \emph{population size},
\emph{number of episodes}, \emph{number of elite} and \emph{mutation rate}.
\paragraph{Population size} In the genetic algorithm modeling of the thesis
the population has the same amount of individuals in each generation. At the
end of the generation, fitness-based selection chooses individuals to produce
offspring using crossover operations. The resulting offspring is then
potentially mutated. The amount of crossover and mutation operations are
influenced by the \emph{population size}. A larger value allows for more
diversity in the population, althouh it is not a guarantee. Diversity is
important when dealing with the possibility of local optima. However, a smaller
value allows for a more directed search towards an optimal solution.
\paragraph{Number of episodes} The number of evaluations that will be assigned
by UCB-1 within a single generation is referred to as the \emph{number of
episodes}. A higher value of this parameter raises the certainty of the
estimated fitness values of each individual. This certainty is important when
the fitness value is subject to noise. UCB-1 however only ranks individuals and
does not express whether more evaluations will be beneficial. If
the fitness of the individuals in the current generation is rather low, a
higher number of episodes will also increase the cumulative regret.
Furthermore, the number of episodes determines the ``duration'' of a single
generation. A lower number of episodes results in more completed generations,
which is important for effective search.
\paragraph{Number of elite}  Elitism, as described in Section~\ref{sec:approach_generation_switch},
preserves the best performing individuals of the previous generation. The
\emph{number of elite} members reduce the number of crossover operations, since
elite members are maintained as is and are not replaced by their offspring.
Moreover elite members are protected against mutation. A higher number of elite
members reduces the influence of randomness on the population. However, elite
members slow down the exploration. A high number of elite members will cause
the algorithm to get stuck in local optima.
\paragraph{Mutation rate} The \emph{mutation rate} determines the chance that a
mutation occurs in the chromosome of a new individual. Mutation ensures that
areas of the search space are explored, even when the genetic algorithm moved
in a different direction. When this parameter value increases, more exploration
occurs. Setting this parameter too high will result in a lack of convergenge.
\section{Simulations}
\label{sec:simulations_simulations}
The parameter values of the genetic algorithm that need to be determined by
simulation are combined in parameter setups. Each setup contains a specific
combination of values for each parameter. The parameter setups are shown in
Table~\ref{tab:simulation_setups}. Two groups of six have been formed out of
these twelve parameter setups. The analysis of the \emph{TutOER} system under
specific parameter setups is done per group. Group~1 contains the first six
parameter setups and covers the interplay between \emph{population size} and
the \emph{number of episodes}. Group~2 contains the second six parameter setups
and covers the interplay between the \emph{number of elite} and the
\emph{mutation rate}. Each parameter setup is labelled with a firstname for
later reference.\\
\begin{table}[h!]
	\centering
	\caption{Parameter setups for the \emph{TutOER} system divided in two groups}
	\label{tab:simulation_setups}
	\begin{tabular}{llllll}\hline
		\textbf{Label} & \textbf{Population size} & \textbf{\# Episodes}
		& \textbf{\# Elite} & \textbf{Mutation rate} \\\hline
		% big names in open education internationally
		\texttt{george} & 5 & 5 & 2 & 0.05 \\
		\texttt{david} & 5 & 10 & 2 & 0.05 \\
		\texttt{stephen} & 5 & 20 & 2 & 0.05 \\
		\texttt{erik} & 10 & 10 & 2 & 0.05 \\
		\texttt{xavier} & 10 & 20 & 2 & 0.05 \\
		\texttt{peter} & 20 & 20 & 2 & 0.05 \\\hdashline
		% Dutch SIG OER members
		\texttt{robert} & 10 & 10 & 0 & 0.05 \\
		\texttt{martijn} & 10 & 10 & 1 & 0.05 \\
		\texttt{ria} & 10 & 10 & 2 & 0.05 \\
		\texttt{ellen} & 10 & 10 & 0 & 0.25 \\
		\texttt{nicolai} & 10 & 10 & 1 & 0.25 \\
		\texttt{saskia} & 10 & 10 & 2 & 0.25 \\
	\end{tabular}
\end{table}\\
\noindent
The parameter setups are tested in two artificial environments. The first
provides evaluation outcomes using the handcrafted model displayed in
Table~\ref{tab:simulation_fitness_model}. The handcrafted model contains
patterns that match to sequences, the explanation of these patterns can be
found in Table~\ref{tab:simulation_fitness_model_explanation}. Sequences that
are not matched by any pattern get a fitness of 0.
The model is based on the author's expectation of the true quality of the
sequences containing resources about the intuition of the game. The first
environment will be referred to as the \emph{normal environment}.\\
\begin{table}[h]
\begin{minipage}[t]{0.48\linewidth}
\caption{Fitness values for each sequence.}
\label{tab:simulation_fitness_model}
\begin{tabular}[ht]{ll}\hline
	Sequence pattern & Fitness \\\hline
	\verb![5]! & 0.7 \\
	\verb![6]! & 0.8 \\
	\verb![7]! & 0.1 \\
	\verb![8]! & 0.8 \\
	\verb!{6, ( 5 | 8 )}! & 1 \\
	\verb!{5, 6, 8}! & 1 \\
	\verb!{5, 6, 7}! & 0.6 \\
	\verb!{*, *, ( 5 | 6 ), ( 5 | 6 )}! & 0.7\\
\end{tabular}
\end{minipage}
\hfill
\begin{minipage}[t]{0.51\linewidth}
\caption{Explanation of sequence patterns.}
\label{tab:simulation_fitness_model_explanation}
\begin{tabular}[ht]{lp{0.6\linewidth}}\hline
	Construct &  Meaning \\\hline
	\verb![5]! & Resource 5.\\
	\verb![*]! & Any resource.\\
	\verb![5, 6]! & First resource 5 followed by resource 6.\\
	\verb!{5, 6}! & Resources 5 and 6 in any order.\\
	\verb![( 5 | 6 )]! & Either resource 5 or 6.\\
	\verb!{7, ( 5 | 6 )}! & Resources 7 and either 5 or 6 in any order.\\
\end{tabular}
\end{minipage}
\end{table}\\
\noindent
The second environment uses the same handcrafted model as the first
environment, but adds gaussian noise of 0.2 standard deviation. The purpose of
the second environment is to see how the performance of the \emph{TutOER}
system under each paramater setup is changed when the observed fitness contains
noise. The second environment will be referred to as the \emph{noisy
environment}.\\\\
\noindent
In all simulations the sequences are restricted in length. Each sequence must
contain one, two or three resources. This restriction is the same as in the
experiment described in Chapter~\ref{ch:experimental_setup}.
\section{General setup}
\label{sec:simulations_setup}
In order to compare the results from simulations more easily, each
simulation is terminated after 1000 evaluations. This will give
a genetic algorithm time to complete 100 generations with 10 evaluations in
each generation. Similarly, a genetic algorithm setup with 20 evaluations in
each generation will only complete half as much generations. The comparison
between the two might not seem fair when looking at the number of generations.
However, the task at hand defines the evaluation to be a scarse resource that
will require a new student each time. It is therefore more interesting to know
what a particular parameter setup accomplishes in a limited amount of
evaluations, rather than anything else. The number of generations that have
been completed after this fixed amount of evaluations is thus a consequence of
the parameter setup. Section~\ref{sec:simulations_parameters} describes the
trade-off between number of generations and number of episodes that follows
from this restriction.
\subsection{Evaluation metrics}
To aid in the analysis of each simulation result, three views are applied to
the data: \emph{the cumulative regret curve}, \emph{the coverage curve} and
\emph{the convergence plot}.
% cumulative regret, number and curve
\paragraph{Cumulative Regret Curve}
Recall that regret is defined as the difference in received reward (i.e.
learning gain) between the presented sequence and the optimal sequence.
The cumulative regret curve shows the built-up of regret after each
evaluation. It provides insight in the handling of exploration versus
exploitation of the system under a particular parameter setup.
Normally, the use of the UCB-1 selection algorithm ensures that
this curve decreases logarithmically over time. However, due to the
combination with the genetic algorithm, the options from which the UCB-1
algorithm can choose are limited. The interplay between the genetic algorithm
and UCB-1 can either boost or reduce performance. This curve visualizes
the amount to which the performance of UCB-1 is disrupted by this interplay.
Each value of the curve is an average of the values of ten repetitions
on the same point. At each point the maximum and minimum value are indicated
using errorbars.
\paragraph{Coverage curve}
The coverage of the system is defined as the percentage of all possible
sequences that was evaluated at least once. The coverage curve shows how this
percentage is built-up per evaluation. In other words, the coverage curve shows
when an evaluation explored uncharted territory of the search space. New
sequences become part of a generation due to crossover, mutation and
immigration. UCB-1 ensures that all sequences that are selectable will be
evaluated at least once\footnote{Provided the number of
episodes is not smaller than the number of individuals in each generation.}.
Therefore, this curve provides an observation of the amount and timing of the
introduction of new individuals in the population. Similar to the cumulative
regret curve, each value is an average of the values of ten repetitions on the
same point. At each point the maximum and minimum value are indicated using
errorbars.
\paragraph{Convergence plot}
The \emph{TutOER} system is said to be converged when ten evaluations in a row
regarded an optimally performing sequence. A sequence is optimal when its true
fitness value is equal to the maximum. In the case of multiple optimal
sequences, an evaluation of any of the optimal sequences counts as the same
optimal evaluation. This could mean that the system is said to have converged
after a series of ten evaluations of different optimal sequences. The
flexibility in this definition is required, because otherwise the system could
be alternating between two equally optimal sequences until termination without
being said to converge. The number of evaluations needed to reach that
convergence point in each repetition is plotted in a boxplot. The fact that
only one datapoint derives from each repetition allows for this detailed look
at the distribution. Parameter setups that ensure a quick convergence are
desired when acquiring evaluations comes with a cost.
\section{Results}
\label{sec:simulations_results}
\subsection{Group 1 (Population size \& \# Episodes)}
\subsubsection*{Analysis}
Figure~\ref{fig:cumul_handcrafted_container_group1} shows the cumulative
regret built-up by the \emph{TutOER} system under the parameter setups of Group~1
in the normal environment. Each simulation setup has
been repeated ten times. Each individual plot shows the averaged cumulative
regret after each evaluation, together with vertical error bars indicating the
maximum and minimum value at every twenty evaluations. The \texttt{stephen} setup stands out with a
much higher cumulative regret built-up than the others. Due to the number of episodes in the
\texttt{stephen} parameter setup, only 50 generations have been completed after
1000 evaluations. Also the \texttt{xavier} and \texttt{peter} setups complete
50 generations. In comparison, the \texttt{george} setup ensures the completion
of 200 generations in the same number of evaluations. The difference between
the number of episodes and the number of individuals in each generation is the
largest for the \texttt{stephen} setups. Recall that UCB-1 selection is applied
to determine which individuals are evaluated. In the \texttt{stephen} setup,
UCB-1 has very few individuals to choose from (namely five) and relatively many
evaluations to assign (namely twenty). This will allow the \emph{TutOER} system
to aquire fitness estimates with much more confidence. However, when none of
the five individuals performs optimal, the \emph{TutOER} system is forced to
to built-up regret until the end of the generation before the genetic algorithm
can continue its search. This is true for all paramater setups, but
because the number of individuals in each generation is relatively low compared
to the number of episodes in the \texttt{stephen} setup, it is more likely to
happen than in a setup where that difference is smaller.\\\\
\noindent
The \texttt{peter}
setup is an example where that difference is much smaller. It has the same
amount of episodes as the \texttt{stephen} setup but a higher number of
individuals in each generation (namely twenty). That allows for more diversity in a
generation. This has no effect in the first generation, because the
population is initialized with sequences of only one resource. Since there are
only four resources in the population, the first generation under the
\texttt{peter} setup will contain sixteen duplicates. From the perspective of
the UCB-1 selection, this has the same effect as in the \texttt{stephen} setup.
However, after the first generation there are more possibilities for
exploration. This is also reflected in the percentage of sequences that have
been observed. Figure~\ref{fig:cover_handcrafted_container_group1}
shows the percentage of possible sequences that have been encountered by the
\emph{TutOER} system under the various parameter setups within Group~1. The
\texttt{peter} setup causes the most exploration. Followed by the other
parameter setups of Group~1 in decreasing order of their \emph{population size}
parameter. However, some of the differences are too close to call. When comparing
parameter setups with the same \emph{population size}, a second
ordering becomes apparent. Parameter setups with a
high number of episodes (e.g. \texttt{stephen}) have encountered a lower percentage of sequences than
parameter setups with a lower number of episodes (e.g. \texttt{george}). This
reverse relation comes from the fact that a higher number of episodes means
the genetic algorithm will complete less generations in the same amount of
evaluations. Completing less generations means less crossover and mutation
applications, and thereby also less opportunities for diversity.\\\\
\noindent
The observation of these orderings in relation to the number of episodes and
the population size also hold in a more noisy environment.
Figure~\ref{fig:cover_handcrafted_noise_container_group1} shows the
percentage of sequences encountered under parameter setups of Group~1 in the
second environment. Noise on the observed fitness values should not
have a direct impact on the level of exploration in the genetic algorithm
setup. What it could affect however is the direction of exploration and the
regret as a result that.
Figure~\ref{fig:cumul_handcrafted_noise_container_group1} shows the
cumulative regret built-up by the \emph{TutOER} system under the parameter
setups of Group~1 in a noisy environment. The differences with the results in
the normal environment are not that big. The parameter setups are slightly more apart in
terms of cumulative regret, but that difference is made early on in the
simulation (Figure~\ref{fig:cumul_handcrafted_noise_overview_group1}).
Given that the lines plotted in the overview are averages of ten repetitions,
there could also be an alternative explanation. For example a mutation may have
occured at a different time causing a different outlier in the simulated
cumulative regrets. This means that there might actually be no effect of the
added noise on the performance of the \emph{TutOER} system under the parameter
setups of Group~1. The impact of the values for the mutation rate and
number of elite in all parameter setups of Group~1 on this noise resillience
will be discussed in Section~\ref{sec:simulation_handcrafted_group2}.\\\\
\noindent
Figure~\ref{fig:conv_group1_handcrafted} shows the boxplot for the ten
repetitions of the number of evaluations needed to converge under each
parameter setup in Group~1. The actual numbers of the convergence at each repetition
can be found in Appendix~\ref{ax_simulation_convergence_data}. The \emph{TutOER} system
converges the fastest on average under the \texttt{erik} parameter setup.
Parameter setups \texttt{george} and \texttt{david} follow in second and third
place, even though they have a better median value. The one that stands out is
the parameter setup \texttt{stephen} with extreme outliers as well as a
relatively high median value. However, apart from one trial that took 769
evaluations to converge, all trials under any parameter setup of Group~1
converged within 100 evaluations. In other words, all simulation setups
resulted in sticking to the optimal sequence of OER for at least ten
consecutive evaluations using less than 100 simulated students. And on average
almost all parameter setups needed less than 50 simulated students to
converge. However, when the number of repetitions is small the outliers matter.
Therefore \texttt{erik} and \texttt{peter} should be favored over
\texttt{david} and \texttt{george}. Figure~\ref{fig:conv_group1_handcrafted_noise}
shows the boxplot distribution of the same parameter setups in a noisy
environment. The same parameter setups have outliers, with roughly the same
number of evaluations that are needed to converge. Parameter setup
\texttt{erik} outperforms the rest, including \texttt{peter}, on convergence in
this noisy environment.
\subsubsection*{Conclusion}
From the plots of cumulative regret, coverage and convergence can
be concluded that \texttt{erik} scores best overal. The \emph{TutOER} system
had on average the lowest cumulative regret under the parameter setup
\texttt{george}, but \texttt{erik} and \texttt{david} followed in second and
third place. The most unique sequences were encountered on average by the
\emph{TutOER} system under \texttt{peter} parameters, followed by the
\texttt{erik} parameters. In terms of convergence, the \texttt{erik} parameters
were most favorable for the \emph{TutOER} system. Overall, the optimal
population size and number of episodes are thus both 10.
\begin{figure}[ht]
	% origin: simulation_2014-05-28-1432
	\begin{subfigure}{0.48\linewidth}
	\includegraphics[width=\linewidth]{images/results/plot_sim_cumul_george_handcrafted.png}
	\caption{\texttt{george} parameters}
	\label{fig:cumul_handcrafted_george}
	\end{subfigure}
	\hfill
	\begin{subfigure}{0.48\linewidth}
	\includegraphics[width=\linewidth]{images/results/plot_sim_cumul_erik_handcrafted.png}
	\caption{\texttt{erik} parameters}
	\label{fig:cumul_handcrafted_erik}
	\end{subfigure}
	\begin{subfigure}{0.48\linewidth}
	\includegraphics[width=\linewidth]{images/results/plot_sim_cumul_david_handcrafted.png}
	\caption{\texttt{david} parameters}
	\label{fig:cumul_handcrafted_david}
	\end{subfigure}
	\hfill
	\begin{subfigure}{0.48\linewidth}
	\includegraphics[width=\linewidth]{images/results/plot_sim_cumul_stephen_handcrafted.png}
	\caption{\texttt{stephen} parameters}
	\label{fig:cumul_handcrafted_stephen}
	\end{subfigure}
	\begin{subfigure}{0.48\linewidth}
	\includegraphics[width=\linewidth]{images/results/plot_sim_cumul_xavier_handcrafted.png}
	\caption{\texttt{xavier} parameters}
	\label{fig:cumul_handcrafted_xavier}
	\end{subfigure}
	\hfill
	\begin{subfigure}{0.48\linewidth}
	\includegraphics[width=\linewidth]{images/results/plot_sim_cumul_peter_handcrafted.png}
	\caption{\texttt{peter} parameters}
	\label{fig:cumul_handcrafted_peter}
	\end{subfigure}
	\begin{subfigure}{\linewidth}
	\includegraphics[width=\linewidth]{images/results/plot_sim_cumul_joined_david_erik_stephen_xavier_peter_george_handcrafted_nobars.png}
	\caption{overview}
	\label{fig:cumul_handcrafted_overview_group1}
	\end{subfigure}
	\caption[Cumulative regret in normal simulated environment for group 1]{Plot with error bars of the cumulative regret of the \emph{TutOER}
	system using specific parameter setups in the normal environment.
	Plot~\ref{fig:cumul_handcrafted_overview_group1} shows the
	different parameter setups in one graph.}
	\label{fig:cumul_handcrafted_container_group1}
\end{figure}

%\begin{figure}[ht]
	% origin: simulation_2014-05-28-1432
%	\centering
%	\includegraphics[width=\linewidth]{images/results/plot_sim_cumul_joined_david_erik_stephen_xavier_peter_george_handcrafted.png}
%	\caption{Cumulative regret plot placeholder}
%	\label{fig:cumul_placeholder1}
%\end{figure}

\begin{figure}[ht]
	% origin: simulation_2014-05-28-2308
%	\begin{subfigure}{\linewidth}
%	\begin{mdframed}[backgroundcolor=nimback, fontcolor=white]
%		\centering
%		Cumulative regret for parameter set 1,in the handcrafted scenario with
%		noise.
%	\end{mdframed}
%	\end{subfigure}
	\begin{subfigure}{0.48\linewidth}
	\includegraphics[width=\linewidth]{images/results/plot_sim_cumul_george_handcrafted_noise.png}
	\caption{\texttt{george} parameters}
	\label{fig:cumul_handcrafted_noise_george}
	\end{subfigure}
	\hfill
	\begin{subfigure}{0.48\linewidth}
		\includegraphics[width=\linewidth]{images/results/plot_sim_cumul_erik_handcrafted_noise.png}
	\caption{\texttt{erik} parameters}
	\label{fig:cumul_handcrafted_noise_erik}
	\end{subfigure}
	\begin{subfigure}{0.48\linewidth}
	\includegraphics[width=\linewidth]{images/results/plot_sim_cumul_david_handcrafted_noise.png}
	\caption{\texttt{david} parameters}
	\label{fig:cumul_handcrafted_noise_david}
	\end{subfigure}
	\hfill
	\begin{subfigure}{0.48\linewidth}
	\includegraphics[width=\linewidth]{images/results/plot_sim_cumul_stephen_handcrafted_noise.png}
	\caption{\texttt{stephen} parameters}
	\label{fig:cumul_handcrafted_noise_stephen}
	\end{subfigure}
	\begin{subfigure}{0.48\linewidth}
	\includegraphics[width=\linewidth]{images/results/plot_sim_cumul_xavier_handcrafted_noise.png}
	\caption{\texttt{xavier} parameters}
	\label{fig:cumul_handcrafted_noise_xavier}
	\end{subfigure}
	\hfill
	\begin{subfigure}{0.48\linewidth}
	\includegraphics[width=\linewidth]{images/results/plot_sim_cumul_peter_handcrafted_noise.png}
	\caption{\texttt{peter} parameters}
	\label{fig:cumul_handcrafted_noise_peter}
	\end{subfigure}
	\begin{subfigure}{\linewidth}
	\includegraphics[width=\linewidth]{images/results/plot_sim_cumul_joined_david_erik_stephen_xavier_peter_george_handcrafted_noise_nobars.png}
	\caption{overview}
	\label{fig:cumul_handcrafted_noise_overview_group1}
	\end{subfigure}
	\caption[Cumulative regret in noisy simulated environment for group 1]{Plot with error bars of the cumulative regret of the \emph{TutOER}
	system using specific parameter setups in the noisy environment.
	Plot~\ref{fig:cumul_handcrafted_noise_overview_group1} shows the
	different parameter setups in one graph.}
	\label{fig:cumul_handcrafted_noise_container_group1}
\end{figure}

%\begin{figure}[ht]
	% origin: simulation_2014-05-28-2308
%	\centering
%	\includegraphics[width=\linewidth]{images/results/plot_sim_cumul_joined_david_erik_stephen_xavier_peter_george_handcrafted_noise.png}
%	\caption{Cumulative regret plot placeholder}
%	\label{fig:cumul_placeholder1}
%\end{figure}

\begin{figure}[ht]
	% origin: simulation_2014-05-29-1301
	\begin{subfigure}{0.48\linewidth}
	\includegraphics[width=\linewidth]{images/results/plot_sim_cover_george_handcrafted.png}
	\caption{\texttt{george} parameters}
	\label{fig:cover_handcrafted_george}
	\end{subfigure}
	\hfill
	\begin{subfigure}{0.48\linewidth}
	\includegraphics[width=\linewidth]{images/results/plot_sim_cover_erik_handcrafted.png}
	\caption{\texttt{erik} parameters}
	\label{fig:cover_handcrafted_erik}
	\end{subfigure}
	\begin{subfigure}{0.48\linewidth}
	\includegraphics[width=\linewidth]{images/results/plot_sim_cover_david_handcrafted.png}
	\caption{\texttt{david} parameters}
	\label{fig:cover_handcrafted_david}
	\end{subfigure}
	\hfill
	\begin{subfigure}{0.48\linewidth}
	\includegraphics[width=\linewidth]{images/results/plot_sim_cover_stephen_handcrafted.png}
	\caption{\texttt{stephen} parameters}
	\label{fig:cover_handcrafted_stephen}
	\end{subfigure}
	\begin{subfigure}{0.48\linewidth}
	\includegraphics[width=\linewidth]{images/results/plot_sim_cover_xavier_handcrafted.png}
	\caption{\texttt{xavier} parameters}
	\label{fig:cover_handcrafted_xavier}
	\end{subfigure}
	\hfill
	\begin{subfigure}{0.48\linewidth}
	\includegraphics[width=\linewidth]{images/results/plot_sim_cover_peter_handcrafted.png}
	\caption{\texttt{peter} parameters}
	\label{fig:cover_handcrafted_peter}
	\end{subfigure}
	\begin{subfigure}{\linewidth}
	\includegraphics[width=\linewidth]{images/results/plot_sim_cover_joined_david_erik_stephen_xavier_peter_george_handcrafted_nobars.png}
	\caption{overview}
	\label{fig:cover_handcrafted_overview_group1}
	\end{subfigure}
	\caption[Percentage seen in normal simulated environment for group 1]{Plot with error bars of the percentage of chromosomes seen by the \emph{TutOER}
	system using specific parameter setups in the normal environment
	Plot~\ref{fig:cover_handcrafted_overview_group1} shows the
	different parameter setups in one graph.}
	\label{fig:cover_handcrafted_container_group1}
\end{figure}


\subsection{Group 2 (\# Elite \& Mutation rate)}
\label{sec:simulation_handcrafted_group2}
\subsubsection*{Analysis}
Figure~\ref{fig:cumul_handcrafted_container_group2} shows the cumulative regret
of the \emph{TutOER} system under the parameter setups of Group~2. A clear
distinction can be made between parameter setups with a low mutation rate and those
with a high mutation rate. The \emph{TutOER} systems builds up less cumulative
regret under \texttt{robert}, \texttt{martijn} and \texttt{ria} parameters than
under their counterparts with higher mutation rates. This can be explained by
the fact that a higher mutation rate typically means more exploration.
Exploration in turn causes higher regret when the newly found sequences turn out to be
less optimal than the current best. The number of elites doesn't seem to affect
the cumulative regret that much in
Figure~\ref{fig:cumul_handcrafted_container_group2}. This is however different
when operating in a noisy environment.
Figure~\ref{fig:cumul_handcrafted_noise_container_group2} shows the cumulative
regret under the parameter setups of Group~2 in the second environment. Observing
the cumulative regret under \texttt{robert} parameters we see that regret rises
significantly from around 100 evaluations.
Figure~\ref{fig:conv_group2_handcrafted_noise} shows the worst convergence result
under the \texttt{robert} parameters is 80 evaluations. That means that the
\emph{TutOER} system already found the optimal sequence but lost it in a
generation transition. Otherwise UCB-1 would have ensured that the
cumulative regret curve continued its logarithmic path, by continuing to select
the optimal sequence. Losing the optimal
sequence in a generation transition can easily be explained by the fact that
the \texttt{robert} parameters set the number of elites to zero. With no elite
preservation, the optimal sequence needs to compete in the fitness-based
roulette wheel selection. In a noisy environment the observed fitness value of
the optimal sequence can be lower than the observed fitness values of less optimal
sequences. Additionally elitism protects the individual from being applied to
crossover or mutation operators. In lack of this protection, the optimal
sequence may have been subject to mutation or crossover and was thereby no
longer present in the generation. The counterpart of \texttt{robert} with a high
mutation rate is \texttt{ellen}. The cumulative regret curve of \texttt{ellen}
is indeed worse than the parameter setups with one or two elites preserved.
However, curiously the difference is not as extreme as with \texttt{robert}.
One would expect that having no elite members would have a bigger impact when
the mutation rate is higher given the larger risk of mutating an already
optimal sequence.\\\\
\noindent
The parameter setup \texttt{ria} has the lowest cumulative regret in both
environments. The \texttt{ria} parameter values are the most conservative of
all. A high number of elites and a low mutation rate keeps the performance of
the \emph{TutOER} system largely unaffected by the added noise in the second
environment. The consequence of a low mutation rate becomes apparent in
Figure~\ref{fig:cover_handcrafted_ria}. The percentage of sequences evaluated by
the system is one of the lowest under the \texttt{ria} parameters.
Other parameter setups with a low mutation rate share the low coverage value.
This is also what you would expect as a result of a low mutation rate. Given
the similarly low cumulative regret under \texttt{ria}, it is not necessarily a
bad thing. Furthermore, the convergence is the best under the \texttt{ria}
parameter setup in both environments.\\\\
\noindent
The parameter setup \texttt{saskia}, which also has a high number of elites,
performs much worse than \texttt{ria}. The cumulative regret under
\texttt{saskia} is a lot higher in comparison, though better than the other
parameter setups with a high mutation rate. The number of evaluations needed to
converge under the \texttt{saskia} setup is comparable to others in its median
value, but several big outliers were the results of \texttt{saskia} in both
environments. The high percentage of sequences that were evaluated under the
\texttt{saskia} values is the other side of that same coin.
\subsubsection*{Conclusion}
From the plots of cumulative regret, coverage and convergence can
be concluded that \texttt{ria} scores best overal. However, this conclusion is
drawn while putting weight on the stability of a low cumulative regret and
conversion points. The coverage of sequences appears to be mostly affected by
a high mutation rate and to a lesser extend by a low number of elite members.
The \emph{TutOER} system performs understandably poorly under the \texttt{ria}
parameter setup. However, better coverage is only favorable when it leads to quicker
convergence and lower regret. In both metrics, \texttt{ria} outperforms the
rest. Albeit with varying margins.

\begin{figure}[ht]
	% origin: simulation_2014-05-28-1622
	\begin{subfigure}{0.48\linewidth}
	\includegraphics[width=\linewidth]{images/results/plot_sim_cumul_robert_handcrafted.png}
	\caption{\texttt{robert} parameters}
	\label{fig:cumul_handcrafted_robert}
	\end{subfigure}
	\hfill
	\begin{subfigure}{0.48\linewidth}
	\includegraphics[width=\linewidth]{images/results/plot_sim_cumul_martijn_handcrafted.png}
	\caption{\texttt{martijn} parameters}
	\label{fig:cumul_handcrafted_martijn}
	\end{subfigure}
	\begin{subfigure}{0.48\linewidth}
	\includegraphics[width=\linewidth]{images/results/plot_sim_cumul_ria_handcrafted.png}
	\caption{\texttt{ria} parameters}
	\label{fig:cumul_handcrafted_ria}
	\end{subfigure}
	\hfill
	\begin{subfigure}{0.48\linewidth}
	\includegraphics[width=\linewidth]{images/results/plot_sim_cumul_ellen_handcrafted.png}
	\caption{\texttt{ellen} parameters}
	\label{fig:cumul_handcrafted_ellen}
	\end{subfigure}
	\begin{subfigure}{0.48\linewidth}
	\includegraphics[width=\linewidth]{images/results/plot_sim_cumul_nicolai_handcrafted.png}
	\caption{\texttt{nicolai} parameters}
	\label{fig:cumul_handcrafted_nicolai}
	\end{subfigure}
	\hfill
	\begin{subfigure}{0.48\linewidth}
	\includegraphics[width=\linewidth]{images/results/plot_sim_cumul_saskia_handcrafted.png}
	\caption{\texttt{saskia} parameters}
	\label{fig:cumul_handcrafted_saskia}
	\end{subfigure}
	\begin{subfigure}{\linewidth}
	\includegraphics[width=\linewidth]{images/results/plot_sim_cumul_joined_robert_saskia_ellen_martijn_nicolai_ria_handcrafted_nobars.png}
	\caption{overview}
	\label{fig:cumul_handcrafted_overview_group2}
	\end{subfigure}
	\caption[Cumulative regret in normal simulated environment for group 2]{Plot with error bars of the cumulative regret of the \emph{TutOER}
	system using specific parameter setups in the normal environment.
	Plot~\ref{fig:cumul_handcrafted_overview_group2} shows the
	different parameter setups in one graph.}
	\label{fig:cumul_handcrafted_container_group2}
\end{figure}

%\begin{figure}[ht]
	% origin: simulation_2014-05-28-1622
%	\centering
%	\includegraphics[width=\linewidth]{images/results/plot_sim_cumul_joined_robert_saskia_ellen_martijn_nicolai_ria_handcrafted.png}
%	\caption{Cumulative regret plot placeholder}
%	\label{fig:cumul_placeholder2}
%\end{figure}

\begin{figure}[ht]
	% origin: simulation_2014-05-28-2118
	\begin{subfigure}{0.48\linewidth}
	\includegraphics[width=\linewidth]{images/results/plot_sim_cumul_robert_handcrafted_noise.png}
	\caption{\texttt{robert} parameters}
	\label{fig:cumul_handcrafted_noise_robert}
	\end{subfigure}
	\hfill
	\begin{subfigure}{0.48\linewidth}
	\includegraphics[width=\linewidth]{images/results/plot_sim_cumul_martijn_handcrafted_noise.png}
	\caption{\texttt{martijn} parameters}
	\label{fig:cumul_handcrafted_noise_martijn}
	\end{subfigure}
	\begin{subfigure}{0.48\linewidth}
	\includegraphics[width=\linewidth]{images/results/plot_sim_cumul_ria_handcrafted_noise.png}
	\caption{\texttt{ria} parameters}
	\label{fig:cumul_handcrafted_noise_ria}
	\end{subfigure}
	\hfill
	\begin{subfigure}{0.48\linewidth}
	\includegraphics[width=\linewidth]{images/results/plot_sim_cumul_ellen_handcrafted_noise.png}
	\caption{\texttt{ellen} parameters}
	\label{fig:cumul_handcrafted_noise_ellen}
	\end{subfigure}
	\begin{subfigure}{0.48\linewidth}
	\includegraphics[width=\linewidth]{images/results/plot_sim_cumul_nicolai_handcrafted_noise.png}
	\caption{\texttt{nicolai} parameters}
	\label{fig:cumul_handcrafted_noise_nicolai}
	\end{subfigure}
	\hfill
	\begin{subfigure}{0.48\linewidth}
	\includegraphics[width=\linewidth]{images/results/plot_sim_cumul_saskia_handcrafted_noise.png}
	\caption{\texttt{saskia} parameters}
	\label{fig:cumul_handcrafted_noise_saskia}
	\end{subfigure}
	\begin{subfigure}{\linewidth}
	\includegraphics[width=\linewidth]{images/results/plot_sim_cumul_joined_robert_saskia_ellen_martijn_nicolai_ria_handcrafted_nobars_noise.png}
	\caption{overview}
	\label{fig:cumul_handcrafted_noise_overview_group2}
	\end{subfigure}
	\caption[Cumulative regret in noisy simulated environment for group 2]{Plot with error bars of the cumulative regret of the \emph{TutOER}
	system using specific parameter setups in the noisy environment.
	Plot~\ref{fig:cumul_handcrafted_noise_overview_group2} shows the
	different parameter setups in one graph.}
	\label{fig:cumul_handcrafted_noise_container_group2}
\end{figure}

%\begin{figure}[ht]
	% origin: simulation_2014-05-28-2118
%	\centering
%	\includegraphics[width=\linewidth]{images/results/plot_sim_cumul_joined_robert_saskia_ellen_martijn_nicolai_ria_handcrafted_noise.png}
%	\caption{Cumulative regret plot placeholder}
%	\label{fig:cumul_noise_placeholder2}
%\end{figure}

\begin{figure}[ht]
	\begin{subfigure}{0.48\linewidth}
	% origin: simulation_2014-05-28-1432
	\includegraphics[width=\linewidth]{images/results/plot_sim_conv_joined_david_erik_stephen_xavier_peter_george_handcrafted.png}
	\caption{Group 1 parameter setups}
	\label{fig:conv_group1_handcrafted}
	\end{subfigure}
	\hfill
	\begin{subfigure}{0.48\linewidth}
	% origin: simulation_2014-05-28-2308
	\includegraphics[width=\linewidth]{images/results/plot_sim_conv_joined_david_erik_stephen_xavier_peter_george_handcrafted_noise.png}
	\caption{Group 1 parameter setups, with noise}
	\label{fig:conv_group1_handcrafted_noise}
	\end{subfigure}
	\begin{subfigure}{0.48\linewidth}
	% origin: simulation_2014-05-28-1622
	\includegraphics[width=\linewidth]{images/results/plot_sim_conv_joined_robert_saskia_ellen_martijn_nicolai_ria_handcrafted.png}
	\caption{Convergence plot}
	\label{fig:conv_group2_handcrafted}
	\end{subfigure}
	\hfill
	\begin{subfigure}{0.48\linewidth}
	% origin: simulation_2014-05-28-2118
	\includegraphics[width=\linewidth]{images/results/plot_sim_conv_joined_robert_saskia_ellen_martijn_nicolai_ria_handcrafted_noise.png}
	\caption{Convergence plot with noise}
	\label{fig:conv_group2_handcrafted_noise}
	\end{subfigure}
	\caption{Convergence plots}
	\label{fig:conv_joined_with_without_placeholder2}
\end{figure}

\begin{figure}[ht]
	% origin: simulation_2014-05-29-1301
	\begin{subfigure}{0.48\linewidth}
	\includegraphics[width=\linewidth]{images/results/plot_sim_cover_robert_handcrafted.png}
	\caption{\texttt{robert} parameters}
	\label{fig:cover_handcrafted_robert}
	\end{subfigure}
	\hfill
	\begin{subfigure}{0.48\linewidth}
	\includegraphics[width=\linewidth]{images/results/plot_sim_cover_martijn_handcrafted.png}
	\caption{\texttt{martijn} parameters}
	\label{fig:cover_handcrafted_martijn}
	\end{subfigure}
	\begin{subfigure}{0.48\linewidth}
	\includegraphics[width=\linewidth]{images/results/plot_sim_cover_ria_handcrafted.png}
	\caption{\texttt{ria} parameters}
	\label{fig:cover_handcrafted_ria}
	\end{subfigure}
	\hfill
	\begin{subfigure}{0.48\linewidth}
	\includegraphics[width=\linewidth]{images/results/plot_sim_cover_ellen_handcrafted.png}
	\caption{\texttt{ellen} parameters}
	\label{fig:cover_handcrafted_ellen}
	\end{subfigure}
	\begin{subfigure}{0.48\linewidth}
	\includegraphics[width=\linewidth]{images/results/plot_sim_cover_nicolai_handcrafted.png}
	\caption{\texttt{nicolai} parameters}
	\label{fig:cover_handcrafted_nicolai}
	\end{subfigure}
	\hfill
	\begin{subfigure}{0.48\linewidth}
	\includegraphics[width=\linewidth]{images/results/plot_sim_cover_saskia_handcrafted.png}
	\caption{\texttt{saskia} parameters}
	\label{fig:cover_handcrafted_saskia}
	\end{subfigure}
	\begin{subfigure}{\linewidth}
	\includegraphics[width=\linewidth]{images/results/plot_sim_cover_joined_robert_saskia_ellen_martijn_nicolai_ria_handcrafted_nobars.png}
	\caption{overview}
	\label{fig:cover_handcrafted_overview_group2}
	\end{subfigure}
	\caption[Percentage chromosomes seen in normal simulated environment for
	group 1]{Plot with error bars of the percentage of chromosomes seen by the \emph{TutOER}
	system using specific parameter setups in the normal environment.
	Plot~\ref{fig:cover_handcrafted_overview_group1} shows the
	different parameter setups in one graph.}
	\label{fig:cover_handcrafted_container_group2}
\end{figure}

\begin{figure}[ht]
	% origin: simulation_2014-05-29-1620
	\begin{subfigure}{0.48\linewidth}
	\includegraphics[width=\linewidth]{images/results/plot_sim_cover_george_handcrafted_noise.png}
	\caption{\texttt{george} parameters}
	\label{fig:cover_handcrafted_noise_george}
	\end{subfigure}
	\hfill
	\begin{subfigure}{0.48\linewidth}
	\includegraphics[width=\linewidth]{images/results/plot_sim_cover_erik_handcrafted_noise.png}
	\caption{\texttt{erik} parameters}
	\label{fig:cover_handcrafted_noise_erik}
	\end{subfigure}
	\begin{subfigure}{0.48\linewidth}
	\includegraphics[width=\linewidth]{images/results/plot_sim_cover_david_handcrafted_noise.png}
	\caption{\texttt{david} parameters}
	\label{fig:cover_handcrafted_noise_david}
	\end{subfigure}
	\hfill
	\begin{subfigure}{0.48\linewidth}
	\includegraphics[width=\linewidth]{images/results/plot_sim_cover_stephen_handcrafted_noise.png}
	\caption{\texttt{stephen} parameters}
	\label{fig:cover_handcrafted_noise_stephen}
	\end{subfigure}
	\begin{subfigure}{0.48\linewidth}
	\includegraphics[width=\linewidth]{images/results/plot_sim_cover_xavier_handcrafted_noise.png}
	\caption{\texttt{xavier} parameters}
	\label{fig:cover_handcrafted_noise_xavier}
	\end{subfigure}
	\hfill
	\begin{subfigure}{0.48\linewidth}
	\includegraphics[width=\linewidth]{images/results/plot_sim_cover_peter_handcrafted_noise.png}
	\caption{\texttt{peter} parameters}
	\label{fig:cover_handcrafted_noise_peter}
	\end{subfigure}
	\begin{subfigure}{\linewidth}
	\includegraphics[width=\linewidth]{images/results/plot_sim_cover_joined_david_erik_stephen_xavier_peter_george_handcrafted_noise_nobars.png}
	\caption{overview}
	\label{fig:cover_handcrafted_noise_overview_group1}
	\end{subfigure}
	\caption[Percentage of chromosomes seen in noisy simulated environment for
	group 1]{Plot with error bars of the percentage of chromosomes seen by the \emph{TutOER}
	system using specific parameter setups in the noisy environment.
	Plot~\ref{fig:cover_handcrafted_noise_overview_group1} shows the
	different parameter setups in one graph.}
	\label{fig:cover_handcrafted_noise_container_group1}
\end{figure}

\begin{figure}[ht]
	% origin: simulation_2014-05-29-1620
	\begin{subfigure}{0.48\linewidth}
	\includegraphics[width=\linewidth]{images/results/plot_sim_cover_robert_handcrafted_noise.png}
	\caption{\texttt{robert} parameters}
	\label{fig:cover_handcrafted_noise_robert}
	\end{subfigure}
	\hfill
	\begin{subfigure}{0.48\linewidth}
	\includegraphics[width=\linewidth]{images/results/plot_sim_cover_martijn_handcrafted_noise.png}
	\caption{\texttt{martijn} parameters}
	\label{fig:cover_handcrafted_noise_martijn}
	\end{subfigure}
	\begin{subfigure}{0.48\linewidth}
	\includegraphics[width=\linewidth]{images/results/plot_sim_cover_ria_handcrafted_noise.png}
	\caption{\texttt{ria} parameters}
	\label{fig:cover_handcrafted_noise_ria}
	\end{subfigure}
	\hfill
	\begin{subfigure}{0.48\linewidth}
	\includegraphics[width=\linewidth]{images/results/plot_sim_cover_ellen_handcrafted_noise.png}
	\caption{\texttt{ellen} parameters}
	\label{fig:cover_handcrafted_noise_ellen}
	\end{subfigure}
	\begin{subfigure}{0.48\linewidth}
	\includegraphics[width=\linewidth]{images/results/plot_sim_cover_nicolai_handcrafted_noise.png}
	\caption{\texttt{nicolai} parameters}
	\label{fig:cover_handcrafted_noise_nicolai}
	\end{subfigure}
	\hfill
	\begin{subfigure}{0.48\linewidth}
	\includegraphics[width=\linewidth]{images/results/plot_sim_cover_saskia_handcrafted_noise.png}
	\caption{\texttt{saskia} parameters}
	\label{fig:cover_handcrafted_noise_saskia}
	\end{subfigure}
	\begin{subfigure}{\linewidth}
	\includegraphics[width=\linewidth]{images/results/plot_sim_cover_joined_robert_saskia_ellen_martijn_nicolai_ria_handcrafted_noise_nobars.png}
	\caption{overview}
	\label{fig:cover_handcrafted_noise_overview_group2}
	\end{subfigure}
	\caption[Percentage chromosomes seen in noisy simulated environment for
	group 1]{Plot with error bars of the percentage of chromosomes seen by the \emph{TutOER}
	system using specific parameter setups in the noisy environment.
	Plot~\ref{fig:cover_handcrafted_noise_overview_group1} shows the
	different parameter setups in one graph.}
	\label{fig:cover_handcrafted_noise_container_group2}
\end{figure}
