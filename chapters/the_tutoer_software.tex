%%%%%%%%%%%%%%%%%%%%%%%%%%%%%%%%
% Chapter: The \textbf{TutOER} Software %
%%%%%%%%%%%%%%%%%%%%%%%%%%%%%%%%
In order to test the approach described in Chapter~\ref{ch:approach}, the
\textbf{TutOER} system was built\footnote{Software can be found on
	\url{https://github.com/mslatour/oertutor}}. \textbf{TutOER} is a web-based tutor
that optimizes the sequence of OER to teach a concept to students.
The two main software modules are the Tutor Module and the Genetic Algorithm
(GA) Module. The latter implements the genetic algorithm approach chosen in
this thesis in order to assist the Tutor module in selecting educational
material. The software is web-based for mainly two reasons. First, it will make
it easier to have people interact with the system, which is important when one
needs to collect large amounts of data. Second, given the inherit intention of
OER to be distributable, most OER are created for a web environment.\\\\
\noindent
This chapter is structured as follows. Section~\ref{sec:software_interface} will
describe the interface of the \textbf{TutOER} system. The Tutor Module is covered in
Section~\ref{sec:software_tutor_module}. In
Section~\ref{sec:software_ga_module}, the implementation of the genetic
algorithm approach in the GA Module will be discussed.
Section~\ref{sec:software_monitor_module} describes the Monitor Module, which
allows for the analysis of the live system. The Logging Module is discussed in
Section~\ref{sec:software_logging_module}. The database schema of the system is
shown in Appendix~\ref{ax_database}.
\section{Interface}
\label{sec:software_interface}
\begin{figure}[ht!]
	\centering
	\includegraphics[width=0.9\linewidth]{images/screen_app_resource.png}
	\caption[Application screenshot - Resource]{Screenshot of the application
	presenting an education resource.}
	\label{fig:screen_app_resource}
\end{figure}
The \textbf{TutOER} software provides an online interface for
students\footnote{The term student is used in the broadest sense: anyone who
wants to learn something} which presents the educational material and
assessments for each knowledge component to the student. The interface can be
seen in Figure~\ref{fig:screen_app_resource} and consist of two main boxes. The
top box indicates how far the student is in the course, which is based on
the knowledge component the student is currently in. The middle box contains
either the educational resource or the test questions.
The middle box also always contains a button through which the user can advance
to the next page. Section~\ref{sec:software_user_flow} describes the user flow
through the system, which determines the result of clicking the button. The
educational content is displayed in an iframe, which means that the content
could also be an independant online resource. Most open educational resources
are of that nature at the moment. That being said, the experiments described in
Chapter~\ref{ch:experimental_setup} only utilize material that was made by the
author and specifically designed to fit within the layout of the \textbf{TutOER}
interface.
\section{Tutor Module}
\label{sec:software_tutor_module}
%TODO: Explain test functionalities
The tutor module contains all program logic and database models related to the
educational task. It handles all interactions with the student and connects
with the genetic algorithm module. This module is responsible for implementing
the user flow, through which the student is guided towards the end of the
course. This flow is described in Section~\ref{sec:software_user_flow}.
\subsection{User Flow}
\label{sec:software_user_flow}
The software enforces a specific flow through the system on the student. This
flow is divided up in phases. The path between the phases is shown in
Figure~\ref{fig:userflow}. The button described in
Section~\ref{sec:software_interface} almost always triggers a change in phase,
as denoted by the arrows in the figure. The rest of this section explains the
different phases and the exact effect of clicking on that button in each phase.
\begin{figure}[ht!]
	\centering
	\includegraphics[width=0.9\linewidth]{images/userflow.pdf}
	\caption[User flow diagram]{Diagram depicting the phases that a user goes
	through and the path between them.}
	\label{fig:userflow}
\end{figure}
\paragraph{New} A student that is new to the system is shown an explanation of
the course. Provided a student does not clear the stored cookie in the browser
or switches browsers altogether, this phase only occurs once in the interaction
between the student and the software. A button is shown to start the course,
which would put the student in the introduction phase.
\paragraph{Introduction} The introduction phase presents the student with the
description of the current knowledge component. The knowledge component is
either the first one, or the knowledge component set in later phases. The
introduction phase is encountered for each knowledge component, provided the
student finishes the course. A button is shown to move to the pre-test phase.
\paragraph{Pre-test} In the pre-test phase the knowledge of the student on the
current knowledge component is assessed. It shows all the questions at the same
time underneath each other on one page. The student is not forced to answer the
questions by form validation or otherwise which refuses to submit a test
without an answer for each question. The button shown at the buttom submits the
answers given to be graded. Based on the score, the student will either move on
to the sequence phase or, if the score is perfect, the student will skip the
current knowledge component. If the knowledge component is skipped, the student
will either be sent to the introduction phase of the next knowledge component
or, if this was the last knowledge component, move forward to the exam phase.
\paragraph{Sequence} The student in the sequence phase is presented the
sequence of educational material that has been selected by the
system\footnote{See Section~\ref{sec:approach_ucb}}. Sequences can contain more
than one educational resource. In that case, only one resource will be shown at
a time, starting with the first of the sequence. A button is displayed which
will bring the student to the next sequence, if there is one. If the student
has reached the end of the sequence, the button will sent the student to the
post-test phase.
\paragraph{Post-test} The post-test phase displays the questions of the
post-test for the current knowledge component. The appearance, the questions
and the button function is identical to the situation in the pre-test phase.
When the answers are graded, the normalized learning gain is calculated to
feed back into the genetic algorithm as fitness value. If there is a next
knowledge component in the course, the student is sent to the introduction
phase of that knowledge component. If this was the last knowledge component,
the student is sent to the exam phase.
\paragraph{Exam} When the student has passed through all phases of all
knowledge components (or skipped them), he or she is sent to the exam phase. In
this phase an exam is presented to the student that needs to be completed
before the student can move on. The exam grade is not used in any way by the
genetic algorithm, but merely provides a way to evaluate the level of the
student after having interacted with the system. When the exam has been
submitted, the student is sent to the done phase.
\paragraph{Done} When all other phased have been completed, a student enters
the last phase. Here a questionaire is presented to the student, which is
optional to fill in. In the experiment there are two types of participants, one
group is coming via Amazon Mechanical Turk and the other through a different
source. The group from Mechinal Turk is shown a button to return to the
Mechinal Turk website to collect their reward, while at the same time
submitting the answers to the questionaire, regardless of whether they are
empty. The other group is shown a button to submit their questionaire answers,
but there is no side-effect. The student remains in this phase and has finished
his or her participation. This is also explained to the student.

\section{GA Module}
\label{sec:software_ga_module}
The genetic algorithm (GA) module is responsible for selecting the sequence of
educational material to be presented to a student. This module is separate from
the tutor module in order to be replaceable by a different approach, as was
already needed earlier in the thesis process when this module replaced its
predecessor that applied a Markov Decision Process. The module implements the
approach described in Chapter~\ref{ch:approach}, but several adjustments were
made to the standard genetic algorithm in order to make it work in a web-based
environment. These adaptations resulted in the web-based genetic algorithm as
described in Section~\ref{sec:web-based_ga}.
\subsection{Web-based genetic algorithm}
\label{sec:web-based_ga}
\begin{figure}[ht]
	\centering
	\includegraphics[width=0.9\linewidth]{images/web-based_genetic_algorithm.pdf}
	\caption[Client-server interaction in web-based adaption of genetic
	algorithm]{Schema of client-server interaction in a web-based adaption of the
	genetic algorithm loop}
	\label{fig:web-based_ga}
\end{figure}
% TODO: Create image to show the non-web version of the Tutor - GA interaction.
% Something with a loop
The implementation of the genetic algorithm had to be adjusted in order to be
applicable in a web context. In particular, the asynchronous nature of the HTTP
protocol requires an adapted step-wise version of the straightforward implementation
using loop constructs.The reader can compare this situation with that of a parallel
implementation where fitness evaluations are run in parallel threads, where in
this analogy the students play the role of the parallel threads. In this parallel
case, it is already necessary to deal with fitness values being sent back from the
threads in a different order than the threads were started in. A difference between
the analogy and the actual situation is that the students are not guaranteed to
provide a fitness value, since they can decide to close the website. That
difference is important because it complicates the decision on when enough
sequences have been evaluated, since you don't know whether a sequence you
assigned to a student for evaluation will actually be evaluated by that
student or whether the system needs to assign it to another student.\\\\
\noindent
The web-based genetic algorithm is best described by its interactions with one
student. The entire list of interactions between the student's browser and the
\textbf{TutOER} software is given by the phases described in Section~\ref{sec:software_user_flow}.
Only the pre-test, sequence and post-test phases are of interest for the web-based
genetic algorithm. Figure~\ref{fig:web-based_ga} shows the relevant parts
of the client-server interaction between the student's browser and the \textbf{TutOER}
software. Note however that these interactions are likely to be interrupted by
interactions with other students. The figure also shows the communication between
the tutor module and the genetic algorithm. There are six numbered components
in the diagram. The rest of the section describes these components in more
detail.

\begin{description}
	\item[1: Initialize population] Before any interaction occurs with students,
		the population of the genetic algorithm is initialized. This is done for
		all populations that are required later on and do not require a
		trigger from the client. As such it is not an interaction, but it has
		been added to this list for completeness.
	\item[2: Group assignment] The pre-test grade is used to determine
		which student group the student will be assigned to. Each student
		group and knowledge component combination is captured in a separate
		population in the genetic algorithm. The assignment of the student to a
		student group, within the context of a knowledge component, determines
		the population from which an individual will be selected to be
		evaluated.
	\item[3: Loop condition check] In a parallel implementation you would start
		each evaluation thread in a loop, iterating for the desired amount of
		episodes. Translating that to this situation would mean that the system
		somehow activates the student to evaluate it. In a web-based
		implementation, everything is client-driven. Nonetheless, the genetic
		algorithm still consist of an, at least implicit, loop for each desired
		episode. This component is designed to bring the two together. It
		checks how many evaluation episodes have been stored in the current
		generation and compares that to the desired total number of episodes.
		If the total has not been reached yet, the UCB-select component is
		executed. If enough evaluations have been collected however, the
		implicit loop has reached its termination condition. This means the
		population must evolve to a new generation, which happens in the
		regerate component.
	\item[4: UCB-select] For the largest part this component is identical to
		what it would be in any other implementation. It uses the UCB-1 formula
		to select which sequence of educational resources it wishes to evaluate,
		while balancing exploration and explotation. Because the evaluation is done
		asynchronously, takes up significant time and is client-driven,
		it is necessary to keep track of which sequences have already been
		selected by UCB and assigned to a student. This in order to prevent
		UCB from unintendedly assigning the same sequence to many students,
		simply because the evaluation results have not come back yet. Therefore
		sequences are locked when they are assigned to a student and UCB can
		only choose from the sequences that are not locked. This is probably
		similar to what you would do in a parallel implementation.

		However, unlike in a typical parallel implementation, it is not at all
		guaranteed that a student will actually study the entire sequence
		and submit the post-test afterwards. When a student decides to stop
		participating for whatever reason, the sequence that was assigned to
		the student would be forever locked. That could result in a situation
		where there are not enough evaluations stored to proceed to a next
		generation, but since all sequences are locked UCB has no way of
		assigning sequences to students. This is solved as follows. First, UCB
		attempts to select a sequence that is not locked. Second, if that is no
		longer possible, the oldest lock is discarded. Third, the method is
		tried again. This could mean that the system wrongly decides to assign
		the sequence to another student. With the consequence that a sequence
		is evaluated more often than UCB chose to. On the other hand, the lock
		could have prevented the UCB in the first place from being able to
		deliberately select the sequence twice. This mechanism is thus choosing
		between two evils, however it will likely not be applied often.

		A side-effect of this is that an evaluation result could actually
		be submitted to the genetic algorithm when it already moved to a new
		generation. The result of this is stored in connection to the new
		generation, which means that effectively a sequence that was not part
		of the new generation could still be evaluated. The sequence could not
		be selected as one of the survivors at the next generation switch, but
		its fitness value is still stored and can be used whenever the sequence
		reappears due to combination or mutation. This seemed to be the best
		solution.
	\item[5: Regenerate] When enough evaluations have been gathered, this
		component executes the generation switch for the current population as
		described in Section~\ref{sec:approach_generation_switch}. The Django
		web framework that was used to implement the \textbf{TutOER} software should
		ensure that during this process the database tables were locked,
		preventing synchronization issues. This was however not tested. After a
		the new generation has been formed, the loop condition check component
		is retried again.
	\item[6: Store evaluation] When a post-test is graded and the normalized
		learning gain is calculated, the resulting fitness value is stored for
		the sequence that was evaluated in the context of the current
		generation. This could be a different generation than the one the
		sequence was selected from.
\end{description}
\section{Monitor Module}
\label{sec:software_monitor_module}
The monitor module provides a real-time insight in the relevant processes,
events and data stored in the database. This module has only been used to
monitor the experiment while it was running and contained three views:
\emph{log}, \emph{student} and \emph{population}. The log view showed a long
list of logged events, as described in Section~\ref{sec:software_logging_module}.
The student view showed a list of the logged events related to a particular
student. It also showed an overview of the test scores, the assigned student
group and the presented sequence for each knowledge component the student
participated in so far. The population view displayed a list of the individuals
in each generation for a specific population. This included observed fitness
values and the cumulative regret graph for that population.
\section{Logging Module}
\label{sec:software_logging_module}
All logging of events is done by the logging module. This module is tighly
connected to the tutor and genetic algorithm modules and provides input to
the monitor module. The tutor and genetic algorithm modules send out signals
when certain events occur, these signals are picked up by the logging module
and stored in the database as log entries. These log entries are then presented
again by the monitor module. The purpose of the logging activities is to aid
monitoring and analysis of what goes on within the \textbf{TutOER} software and its
modules. The following events are stored in the log entry table.
\begin{leftbar}
\begin{description}
	\item[Error] Any unexpected errors that were caught in the system are
		logged by the module.
	\item[New participant from external source] The module logs each new
		participant that came from an external source. In particularly it
		stores the identifiers of people participating via Amazon
		Mechanical Turk, in order to be able to reconstruct the connection
		between the student object in the database and the Mechanical Turk
		user in case something went wrong. This was especially applicable
		during the experiment period.
	\item[Change in student state] When a student object is created in the
		system en each time the student enters a new phase in the user
		flow, the module logs this.
	\item[Trial created] A trial is the particular instance of the tutor
		teaching actions for a specific knowledge component and a specific
		student. The trial object contains the pre-test and post-test
		results and the sequence that was presented to the student. The
		module logs when this object is created it.
	\item[New generation in population] Each time a population moves to a
		new generation, this is logged by the module.
	\item[New immigrant in the population] The module logs each time an
		individual immigrates in a population. The individual that the
		immigrant replaced is also stored in the log entry.
\end{description}
\end{leftbar}
\section{Bootstrap values after restart}
The experiment had to be restarted after 492 evaluations were already gathered.
This was due to some technical changes that had to be made. No changes were
made to the resources or the assessments. As such the gathered evaluations
would still be valid observations. In order to reuse these evaluations at least
partly, the \textbf{TutOER} system was equiped with a bootstrap mechanism. This
mechanism reuses evaluations from earlier experiments where possible.\\\\
\noindent
The technical changes affected the search behavior of the genetic algorithm.
It is therefore not likely that all evaluations will be reusable. In fact only
when UCB-1 selects a particular sequence to be evaluated, that evaluation could
be bootstrapped. The bootstrap procedure extends step 4 in
Figure~\ref{fig:web-based_ga} and goes as follows.
\begin{enumerate}
	\item Retrieve UCB-1 selection
	\item If selected sequence is present in unused bootstrap values,
		\begin{enumerate}
			\item Submit bootstrap value as if a student evaluated it at that moment
			\item Return to step 1.
		\end{enumerate}
	\item Else, present sequence to student.
\end{enumerate}
