%%%%%%%%%%%%%%%%%%%%%%%%%
% Chapter: Related Work %
%%%%%%%%%%%%%%%%%%%%%%%%%
%\section{Recommender Systems}
%\section{Intelligent Tutoring Systems}
%\section{Adaptive Hypermedia Systems}
%\section{Genetic algorithms}
%There are many different implementations of fitness functions for the
%curriculum sequencing problem~\citep{AlMuhaideb2011}.
%Several systems include a term that expresses how
%well a particular solution fits the pre-determined prerequisite structure of
%the learning objects~\citep{Seki2005, Chen2009, Samia2007}.
%
%The approach presented here takes a different
%approach. A big part of the prerequisite structure is already captured by the
%given main curriculum in this thesis, and as such already provides the order of
%knowledge components. In this thesis, the curriculum sequencing step takes place
%on the lower level of presentation sequencing, where the different learning
%objects all try to convey the same content but do so in different
%ways. Some objects might contain an example or the formal description. Some
%might present the knowledge in text, others are more visual. Different students
%require a different mix of presentations and potentially at different
%orders\footnote{citation needed}.
%
%Other work in this field includes a term that expresses how smooth the
%transitions are between learning objects in difficulty~\citep{Hovakimyan2004,
%Seki2005, Chen2008, Huang2007} or how well their difficulty matches with the
%compentency level of the student\citep{Seki2005, Chen2008, Chen2009, Samia2007, Huang2007}.


